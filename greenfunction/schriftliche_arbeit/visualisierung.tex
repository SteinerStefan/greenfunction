\section{Visualisierung}
	
	Wir speichern die erhaltene Matrix in einem .csv File mit einer Genauigkeit von vier Nachkommastellen ab, was absolut hinreichen f�r eine Visualisierung ist. Anfangs benutzten wir \verb|MATLAB| um aus dem CSV-File eine anst�ndige Visualisierung zu erhalten. Der Aufwand war gering und die Resultate waren schnell kontrollierbar.
	
	Um das zu automatisieren, benutzen wir sp�ter Gnuplot, welches sich mit C-Code ohne Probleme ansprechen l�sst und uns die Bilder automatisch mit dem vorgegebenen Einstellungen erstellt.
	
	Gnuplot ist ein Kommandozeilen basiertes OpenSource Tool, das aber auch mit GUI erh�ltlich ist. �ber eine Pipe lassen sich lassen sich alle Einstellungen vornehmen und aus dem .CSV-File ein PNG generieren. 
	
	Mit der Grenn-Funtion k�nnen bei einer partiellen DGL 2. Ordnung einzelne Punkte berechnet werden. Um das zu visualisieren, berechnen wir einzelne Ringe bzw. Quadrate vom Zentrum ausgehen nach aussen. Man k�nnte nat�rlich auch jeder Punkt einzeln dazu nehmen, was mehr Berechnungen erfordert.
	
	