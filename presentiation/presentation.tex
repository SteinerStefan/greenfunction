\documentclass{beamer}
\usepackage[T1]{fontenc}
\usepackage[latin1]{inputenc}
\usepackage[ngerman]{babel}

\usetheme{Dresden}
\usecolortheme{beaver}
\setbeamercovered{transparent}
\beamertemplatenavigationsymbolsempty
\setbeamertemplate{footline}[frame number]

\title{Proseminar Grenzen der Berechenbarkeit}

\author[M. Schulz]{
	Michael Schulz
}

\begin{document}

	\frame{
		\titlepage
	}

	\begin{frame}
		Inhalt
	\end{frame}
	
	\begin{frame}[fragile]{"Uberschrift}
		\begin{verbatim}
			int foo() { return 0; }
		\end{verbatim}
	\end{frame}

	\frame[<+->][label=Liste]{
		\frametitle{"Uberschrift}
		\begin{itemize}
		\item uno
		\item duo
		\end{itemize}
	}

	\frame{
		\frametitle{Inhaltsverzeichnis}
		\tableofcontents
		[pausesections]
	}
	
	\begin{columns}
	        \column{.55\textwidth}
	                \pgfimage[width=\textwidth]{schaubild}
	        \column{.45\textwidth}
	                \begin{enumerate}
	                \item Start
	                \item Stopp
	                \end{enumerate}
	                
	        
	\end{columns}
	
	\begin{frame}
		\begin{block}{Blocktitel}
		        Blocktext
		\end{block}
		
		\begin{exampleblock}{Beispielblocktitel}
		        Beispielblocktext
		\end{exampleblock}
		
		\begin{alertblock}{Warnungsblocktitel}
			                Warnungsblocktext
		\end{alertblock}
	\end{frame}
	
	\begin{frame}
		\begin{itemize}
		        \item Einleitung
		        \item<2-> daher
		        \item<alert@3> aber Achtung!
		        \item<3-> also so und so
		        \item<4-> Schlussfolgerung
		\end{itemize}
	\end{frame}
	
\end{document}