% Dieser Text ist urheberrechtlich gesch\"utzt
% Er stellt einen Auszug eines von mir erstellten Referates da
% und darf nicht gewerblich genutzt werden
% die private bzw. Studiums bezogen Nutzung ist frei
% April 2011
% Autor: Sascha Frank 
% Universit\"at Freiburg 
% www.informatik.uni-freiburg.de/~frank/
% frank < was da sonst immer steht > tf.uni-freiburg.de


\documentclass[hyperref={pdfpagelabels=false}]{beamer}
% Die Hyperref Option hyperref={pdfpagelabels=false} verhindert die Warnung:
% Package hyperref Warning: Option `pdfpagelabels' is turned off
% (hyperref)                because \thepage is undefined. 
% Hyperref stopped early 
%

\usepackage{lmodern}
% Das Paket lmodern erspart die folgenden Warnungen:
% LaTeX Font Warning: Font shape `OT1/cmss/m/n' in size <4> not available
% (Font)              size <5> substituted on input line 22.
% LaTeX Font Warning: Size substitutions with differences
% (Font)              up to 1.0pt have occurred.
%

% Wenn \titel{\ldots} \author{\ldots} erst nach \begin{document} kommen,
% kommt folgende Warnung:
% Package hyperref Warning: Option `pdfauthor' has already been used,
% (hyperref) ... 
% Daher steht es hier vor \begin{document}

\title{Beamer Class etwas netter}   
\author{Sascha Frank} 
\date{\today} 

% zusaetzlich ist das usepackage{beamerthemeshadow} eingebunden 
\usepackage{beamerthemeshadow}
\begin{document}


\begin{frame}
\titlepage
\end{frame} 

\section{Abschnitt Nr.1} 
\begin{frame}
\frametitle{Titel} 
Die einzelnen Frames sollte einen Titel haben 
\end{frame}

\subsection{Unterabschnitt Nr.1.1  }
\begin{frame} 
Denn ohne Titel fehlt ihnen was
\end{frame}

\begin{frame}
\frametitle{Aufz\"ahlung mit Pausen}
\begin{itemize}[<+->]
\item  Einf\"uhrungskurs in \LaTeX{} 
\item  Kurs 2 
\item  Seminararbeiten und Pr\"asentationen mit \LaTeX{} 
\item  Die Beamerclass
\end{itemize} 
\end{frame}

	\begin{frame}
		Inhalt
	\end{frame}
	
	\begin{frame}[fragile]{"Uberschrift}
		\begin{verbatim}
			int foo() { return 0; }
		\end{verbatim}
	\end{frame}

	\frame[<+->][label=Liste]{
		\frametitle{"Uberschrift}
		\begin{itemize}
		\item uno
		\item duo
		\end{itemize}
	}

	\frame{
		\frametitle{Inhaltsverzeichnis}
		\tableofcontents
		[pausesections]
	}
	
	\begin{columns}
	        \column{.55\textwidth}
	                \pgfimage[width=\textwidth]{schaubild}
	        \column{.45\textwidth}
	                \begin{enumerate}
	                \item Start
	                \item Stopp
	                \end{enumerate}
	                
	        
	\end{columns}
	
	\begin{frame}
		\begin{block}{Blocktitel}
		        Blocktext
		\end{block}
		
		\begin{exampleblock}{Beispielblocktitel}
		        Beispielblocktext
		\end{exampleblock}
		
		\begin{alertblock}{Warnungsblocktitel}
			                Warnungsblocktext
		\end{alertblock}
	\end{frame}
	
	\begin{frame}
		\begin{itemize}
		        \item Einleitung
		        \item<2-> daher
		        \item<alert@3> aber Achtung!
		        \item<3-> also so und so
		        \item<4-> Schlussfolgerung
		\end{itemize}
	\end{frame}

\end{document}
