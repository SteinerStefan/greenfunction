\section{FITS Dateiformat}\label{sec:FITS}

Flexible Image Transport System ist die ausgeschriebene Bezeichnung f�r das Wort FITS.Es ist ein quelloffenes Dateiformat, womit Daten verlustlos abgespeichert werden k�nnen. Die Nasa entwickelte es und in der Astronomie hat es sich als Standardformat etabliert.  Informationen k�nnen in verschiedenen Layern abgelegt werden. Das bedeutet, ein Gitterpunkt in einem Datenfeld kann unterschiedliche Informationen enthalten. Angenehm ist ausserdem, dass die Werte direkt in Flisskommadarstellung abgespeichert werden k�nnen.\cite{nasa:fits}

Wir haben es in diesem Projekt als Eingabebild verwendet, weil wir mit einem einfachen Bildbearbeitungsprogramm Testbilder herstellen konnten.  
