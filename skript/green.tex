\section{Der Hintergrund �ber die Greensche Funktion}
Der Namensgeber f�r die Greensche Funktion ist George Green (1793-1841). Er war ein britischer Physiker und Mathematiker. Sein Vater betrieb eine M�hle und George Green arbeitete ebenfalls als M�ller. Nach dem Tod seines Vaters f�hrte er den M�hlenbetrieb fort. Bemerkenswert ist, dass Green nur zwei Jahre in die Schule ging. Mathematische und Physikalische Grundlagen brachte er sich selber bei. Der Ort an dem er lehrte war seine M�hle. Da nie ein Portrait von ihm angefertigt wurde, gibt es kein Bild von ihm. Darum wird anstatt seinem Konterfei jeweils eine Windm�hle verwendet um ihn darzustellen. Die Windm�hle gibt es �brigens immer noch. Green ver�ffentliche mit 35 Jahren seine erste Arbeit. Diese wurde kaum beachtet ausser von einem adligen Namens Sir Edward Bromhead. Durch ihn konnte George Green im Alter von vierzig Jahren anfangen in Cambridge studieren. Vier Jahre nachdem er graduierte starb er jedoch an einer schweren Grippe. So geriet seine Arbeit zu unrecht f�r einige Jahre in Vergessenheit. Seine Arbeit wurde jedoch einige 


Quellen: \\
http://www.math.washington.edu/~morrow/334\_13/green.pdf, aufgerufen: 13.05.2014\\
http://en.wikipedia.org/wiki/George\_Green, aufgerufen: 14.05.2014 
