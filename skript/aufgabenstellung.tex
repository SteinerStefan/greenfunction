\section{Aufgabenstellung}
Um ein Problem der Mathematik besser zu verstehen, ist es oft sehr hilfreich, wenn es verst�ndlich beschrieben ist. Eine Visualisierung in Form eines Bildes oder kurzen Films ist dabei eine M�glichkeit dies zu bewerkstelligen.

Die Hauptaufgabe dieses Projektes war es nun, die Greensche Funktion in zwei Dimensionen zu visualisieren. Die L�sung f�r eine Dimension wurde bereits in einem kurzen Film veranschaulicht \footnote{\url{https://www.youtube.com/watch?v=Wpi7Gf7V2HY}}. Es soll dabei zuerst eine partielle Differentialgleichung mithilfe 
%des Greenschen Ansatzes 
gel�st und danach das visualisieren umgesetzt werden. Die Greensche Funktion erm�glicht die Berechnung eines Problems durch Supperposition. Darum sollten auch verschiedene Anfangswerte untersucht werden und wie diese sich auf die L�sungen auswirken. 

Als Differentialgleichung sei dabei ein Potentialproblem vorgegeben. Vorliegend ist eine leitende Platte, die am Rande geerdet sei. Wenn nun ein Potential an einem oder mehreren Punkten auf dieser Platte anliegt, ist es von Interesse zu wissen, welches Potential man nun an einem beliebigen Punkt auf der Platte misst. 
%\begin{equation}
%	\dfrac{\partial^2 u}{\partial x^2}+\dfrac{\partial^2 u}{\partial y^2} = f(x,y).
%\end{equation}
%Dabei sei hier noch auf die Arbeit von Reto Christen und Philip Solenthaler verwiesen. Diese haben das Potentialproblem genauer untersucht. 
