\usepackage[ngerman]{varioref}
\usepackage[german]{fancyref}
\usepackage{graphicx,amssymb,amsmath,relsize}
%--------------------------------------------- Kopf- und Fusszeile
\usepackage{scrpage2}
\pagestyle{scrheadings}
\clearscrheadings
\clearscrplain
\ifoot{\headfootnotefont\authorinfo}
%------ % SeitenZahlen outer foot
\usepackage{lastpage}
\ofoot{\headfootnotefont Seite \thepage { }von \pageref{LastPage}}
%------
\ohead{\headfootnotefont\oheaderinfo}
\ihead{\headfootnotefont\iheaderinfo}
\chead{\headfootnotefont\cheaderinfo}
\setheadsepline{.4pt} 											% Linie unter der Kopfzeile
%--------------------------------------------- TikZ und GNU Plot
\usepackage{tikz}
%---------------------------------------------
\usepackage{hyperref}
%--------------------------------------------- Tabellen
\usepackage{array}
\usepackage{booktabs}
\usepackage{multirow}
\usepackage{multicol}
\newcommand{\tabH}[1]{\parbox[1pt][#1em][c]{0cm}{}}
\newcommand{\red}[1]{\textcolor{red}{#1}}
\newcommand{\blue}[1]{\textcolor{blue}{#1}}
\newcommand{\orange}[1]{\textcolor{orange}{#1}}

%------------------- C-Code ------------------- %
\definecolor{colorkeywordstyle}{RGB}{127,0,127}
\definecolor{colorcommentstyle}{RGB}{0,116,0}
\definecolor{colorstringstyle}{RGB}{196,26,22}
\definecolor{colorbackground}{RGB}{240,240,240}
\definecolor{royalblue}{rgb}{0.15,0.25,0.55}
\definecolor{black}{rgb}{0,0,0}

\usepackage{listings}
\lstnewenvironment{code}[1][]
  {\lstset{
  	backgroundcolor =		\color{colorbackground},
    language =				C++,			
    basicstyle =    		\ttfamily\small,
    breaklines = 			true,
    frame = 				none,
    columns =				flexible,
    keepspaces = 			true,
    keywordstyle=			\color{colorkeywordstyle},
    identifierstyle	=\color{royalblue},
    showstringspaces =		false,
    extendedchars =			true,   
    commentstyle = 			\color{colorcommentstyle},,
    numbers = 				left,
    numbersep = 			2pt,
    numberstyle =			\tiny,
    rulecolor=\color{black}, 
    stepnumber =			1,
    tabsize = 				4,
  }}{\vspace{0em}}
%---------------------------------------------
\newcommand{\dunderline}[1]{\underline{\underline{#1}}}