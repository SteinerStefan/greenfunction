%\usepackage[	left	=1.5cm, %
%				right	=1cm, %
%				top		=1cm, %
%				bottom	=1cm, %
%includeheadfoot]{geometry}

\usepackage[ngerman]{varioref}
\usepackage[german]{fancyref}
\usepackage{graphicx,amssymb,amsmath}
\usepackage{framed}						% F�r Graue Beispielboxen
\usepackage{epstopdf}					% Zum einbinden von .eps Dateien (z.B. Matlab)
\usepackage{stmaryrd}					% Dreieck-Sternschaltung
%--------------------------------------------- Dsrstellung der Einheiten
\usepackage[locale=DE]{siunitx}
\sisetup{per-mode=symbol}
%--------------------------------------------- Kopf- und Fusszeile

\usepackage{scrpage2}
\pagestyle{scrheadings}
\clearscrheadings
\clearscrplain
\ifoot{\headfootnotefont\authorinfo}
%------ % SeitenZahlen outer foot
\usepackage{lastpage}
\ofoot{\headfootnotefont Seite \thepage { }von \pageref{LastPage}}
%------
\ohead{\headfootnotefont\oheaderinfo}
\ihead{\headfootnotefont\iheaderinfo}
\chead{\headfootnotefont\cheaderinfo}
\setheadsepline{.4pt} 											% Linie unter der Kopfzeile

%--------------------------------------------- TikZ und GNU Plot
\usepackage{tikz}
\usetikzlibrary{arrows,shapes,trees,calc,patterns}
\usepgflibrary{arrows}
\usepackage{pgfplots}
\usepackage[europeanresistors,europeaninductors]{circuitikz} % F�r Schaltelemente

\pgfdeclarelayer{bg}    % declare background layer
\pgfsetlayers{bg,main}
\newcommand*{\rechterWinkelRadius}{.25cm}
\newcommand*{\rechterWinkel}[2]{% #1 = point, #2 = start angle
   \draw[shift={(#2:\rechterWinkelRadius)}] (#1) arc[start angle=#2, delta angle=90, radius = \rechterWinkelRadius];
   \fill[shift={(#2+45:\rechterWinkelRadius/2)}] (#1) circle[radius=1.25\pgflinewidth];
}


%---------------------------------------------




%--------------------------------------------- & Eigene Liste mysec
\usepackage{hyperref}
\usepackage{tocloft}

%\newlistof{mytopsec}{sec}{\nameoflist}
%\newcommand{\mytopsec}[1]{%
%\refstepcounter{mytopsec}
%\par\noindent{\textsf{\textbf{\themytopsec.\hspace{0.5em} #1}}}
%\addcontentsline{sec}{mytopsec}{\protect\numberline{\themytopsec}#1}\par}
%
%

\newlistof{mysec}{sec}{\nameoflist}
\newcommand{\mysec}[1]{%
\refstepcounter{mysec}
\par\noindent{\textsf{\textbf{\themysec.\hspace{0.5em} #1}}}
\addcontentsline{sec}{mysec}{\protect\numberline{\themysec}#1}\par}
\newlistentry[mysec]{mysubsec}{sec}{1}
\newcommand{\mysubsec}[1]{%
\refstepcounter{mysubsec}
\par\noindent{\textsf{\themysubsec\hspace{0.5em}#1}}
\addcontentsline{sec}{mysubsec}{\protect\numberline{\themysubsec}#1}}
\setcounter{secdepth}{2}
\newlistentry[mysubsec]{mysubsubsec}{sec}{2}
\newcommand{\mysubsubsec}[1]{%
\refstepcounter{mysubsubsec}
\par\noindent{\textsf{\themysubsubsec\hspace{0.5em}#1}}
\addcontentsline{sec}{mysubsubsec}{\protect\numberline{\themysubsubsec}#1}}
\setcounter{secdepth}{3}

%--------------------------------------------- Tabellen

\usepackage{array}
\usepackage{tabularx}
\usepackage{siunitx}
\usepackage{booktabs}
\usepackage{multirow}
\usepackage{multicol}
\newcommand{\tabH}[1]{\parbox[1pt][#1em][c]{0cm}{}}
\newcommand{\red}[1]{\textcolor{red}{#1}}
\newcommand{\blue}[1]{\textcolor{blue}{#1}}
\newcommand{\orange}[1]{\textcolor{orange}{#1}}

\newcommand{\schaum}[1]{\textcolor{red!50}{\scriptsize Schaum S.#1}}
\newcommand{\buch}[1]{\textcolor{red!70!white}{\scriptsize Buch Seite #1}}
\newcommand{\skript}[1]{\textcolor{red!70!white}{\scriptsize Skript Seite #1}}
\newcommand{\uZ}{\underline{Z}}

%-------------- Eigene Befehle, nur tabularx Umgebung
\newcolumntype{Y}{>{\small\raggedright\arraybackslash}X}
\newcolumntype{m}{>{\parbox[1pt][2em][c]{0cm}{}\centering\arraybackslash}X}
\newcolumntype{n}{>{\parbox[1pt][2.5em][c]{0cm}{}\centering\arraybackslash}X}
\newcolumntype{M}{>{\parbox[1pt][3em][c]{0cm}{}\centering\arraybackslash}X}
\newcolumntype{B}{>{\parbox[1pt][4em][c]{0cm}{}\centering\arraybackslash}X}

%--------------------------------------------- Mathe packages

\usepackage{relsize}					% f�r grosse Summen- und Integralzeichen \mathlarger\int\limits_{min}^{max}
\usepackage{amsmath}
\DeclareMathOperator{\arccot}{arccot}
\DeclareMathOperator{\arsinh}{arsinh}
\DeclareMathOperator{\arcosh}{arcosh}
\DeclareMathOperator{\artanh}{artanh}
\DeclareMathOperator{\arcoth}{arcoth}
\DeclareMathOperator{\cjs}{cjs}
\DeclareMathOperator{\IM}{Im}
\DeclareMathOperator{\RE}{Re}
\DeclareMathOperator{\Ln}{Ln}
\DeclareMathOperator{\sgn}{sgn}
\DeclareMathOperator{\Var}{Var}
\DeclareMathOperator{\var}{var}
\DeclareMathOperator{\cov}{cov}
\DeclareMathOperator{\sinc}{sinc}

%\newcommand{\korrpaar}{\hspace{1em}\circ -\hspace{0.8mm} \bullet\hspace{1em}}
\unitlength1cm
% Zeitbereich -- Frequenzbereich
\newcommand{\korrpaar}
{
\begin{picture}(1,0.5)
\put(0.2,0.1){\circle{0.14}}\put(0.27,0.1){\line(1,0){0.5}}\put(0.77,0.1){\circle*{0.14}}
\end{picture}
}
% Frequenzbereich -- Zeitbereich
\newcommand{\Korrpaar}
{
\begin{picture}(1,0.5)
\put(0.2,0.1){\circle*{0.14}}\put(0.27,0.1){\line(1,0){0.45}}\put(0.80,0.1){\circle{0.14}}
\end{picture}
}
%--------------------------------------------- Nummerierung von Gleichungen

%Setzt den equation-Zaehler nach jeder Seite zurueck
\numberwithin{equation}{section}	

%Definiert den Stil:
\renewcommand{\theequation}{\arabic{section}.\arabic{equation}}

%--------------------------------------------- Graue Beispielboxen

\definecolor{MyBoxColor}{rgb}{0.9,0.9,0.9}
\newenvironment{shadedSmaller}{
  \def\FrameCommand{\fboxsep=\FrameSep \colorbox{MyBoxColor}}
  \MakeFramed {\advance\hsize-2\width\FrameRestore}}
{\endMakeFramed}

\newenvironment{Beispiel}{
  \def\FrameCommand{\fboxsep=1em \colorbox{MyBoxColor}}
  \MakeFramed {\advance\hsize-1.1\width\FrameRestore}}
{\endMakeFramed}

\newcommand{\loesung}{\subsubsection*{L�sung}}

\newenvironment{teilaufgaben}{
\begin{enumerate}
\renewcommand{\labelenumi}{\alph{enumi})}
}{\end{enumerate}}

%---------------------------------------------