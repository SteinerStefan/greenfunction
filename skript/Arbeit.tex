\documentclass[a4paper, parskip=half, DIV15]{scrartcl}
\usepackage[T1]{fontenc}
\usepackage[latin1]{inputenc}
\usepackage[ngerman]{babel}
%-------------- Kopf- und Fusszeile
\newcommand{\authorinfo}{Andreas Linggi, Stefan Steiner}
% �usserer Kopf
		\newcommand{\oheaderinfo}{Mathematisches Seminar}
% innerer Kopf
		\newcommand{\iheaderinfo}{}
% zentrierter Kopf
		\newcommand{\cheaderinfo}{}
%------- Rand falls eingeschaltet
%\newcommand{\randabstand}{1.5cm}
%--------------------------------------------- Schriften
%------- Schriften Kopf- und Fusszeile
\newcommand{\headfootnotefont}{\scriptsize\upshape}
%------- Schrift von Listenelementen
\newcommand{\listfont}{\sffamily}
%--------------------------------------------- Name von Eigener Liste
\newcommand{\nameoflist}{\oheaderinfo \hspace{1cm}\small{\authorinfo} \hspace{3cm}\today}
%--------------------------------------------- Masse Tabellen
\newcommand{\tabellenbreite}{14cm}
\newcommand{\notizenbreite}{4cm}
\newcommand{\vollebreite}{17.5cm}
%---------------------------------------------


%\usepackage[	left	=1.5cm, %
%				right	=1cm, %
%				top		=1cm, %
%				bottom	=1cm, %
%includeheadfoot]{geometry}

\usepackage[ngerman]{varioref}
\usepackage[german]{fancyref}
\usepackage{graphicx,amssymb,amsmath}
\usepackage{framed}						% F�r Graue Beispielboxen
\usepackage{epstopdf}					% Zum einbinden von .eps Dateien (z.B. Matlab)
\usepackage{stmaryrd}					% Dreieck-Sternschaltung
%--------------------------------------------- Dsrstellung der Einheiten
\usepackage[locale=DE]{siunitx}
\sisetup{per-mode=symbol}
%--------------------------------------------- Kopf- und Fusszeile

\usepackage{scrpage2}
\pagestyle{scrheadings}
\clearscrheadings
\clearscrplain
\ifoot{\headfootnotefont\authorinfo}
%------ % SeitenZahlen outer foot
\usepackage{lastpage}
\ofoot{\headfootnotefont Seite \thepage { }von \pageref{LastPage}}
%------
\ohead{\headfootnotefont\oheaderinfo}
\ihead{\headfootnotefont\iheaderinfo}
\chead{\headfootnotefont\cheaderinfo}
\setheadsepline{.4pt} 											% Linie unter der Kopfzeile

%--------------------------------------------- TikZ und GNU Plot
\usepackage{tikz}
\usetikzlibrary{arrows,shapes,trees,calc,patterns}
\usepgflibrary{arrows}
\usepackage{pgfplots}
\usepackage[europeanresistors,europeaninductors]{circuitikz} % F�r Schaltelemente

\pgfdeclarelayer{bg}    % declare background layer
\pgfsetlayers{bg,main}
\newcommand*{\rechterWinkelRadius}{.25cm}
\newcommand*{\rechterWinkel}[2]{% #1 = point, #2 = start angle
   \draw[shift={(#2:\rechterWinkelRadius)}] (#1) arc[start angle=#2, delta angle=90, radius = \rechterWinkelRadius];
   \fill[shift={(#2+45:\rechterWinkelRadius/2)}] (#1) circle[radius=1.25\pgflinewidth];
}


%---------------------------------------------




%--------------------------------------------- & Eigene Liste mysec
\usepackage{hyperref}
\usepackage{tocloft}

%\newlistof{mytopsec}{sec}{\nameoflist}
%\newcommand{\mytopsec}[1]{%
%\refstepcounter{mytopsec}
%\par\noindent{\textsf{\textbf{\themytopsec.\hspace{0.5em} #1}}}
%\addcontentsline{sec}{mytopsec}{\protect\numberline{\themytopsec}#1}\par}
%
%

\newlistof{mysec}{sec}{\nameoflist}
\newcommand{\mysec}[1]{%
\refstepcounter{mysec}
\par\noindent{\textsf{\textbf{\themysec.\hspace{0.5em} #1}}}
\addcontentsline{sec}{mysec}{\protect\numberline{\themysec}#1}\par}
\newlistentry[mysec]{mysubsec}{sec}{1}
\newcommand{\mysubsec}[1]{%
\refstepcounter{mysubsec}
\par\noindent{\textsf{\themysubsec\hspace{0.5em}#1}}
\addcontentsline{sec}{mysubsec}{\protect\numberline{\themysubsec}#1}}
\setcounter{secdepth}{2}
\newlistentry[mysubsec]{mysubsubsec}{sec}{2}
\newcommand{\mysubsubsec}[1]{%
\refstepcounter{mysubsubsec}
\par\noindent{\textsf{\themysubsubsec\hspace{0.5em}#1}}
\addcontentsline{sec}{mysubsubsec}{\protect\numberline{\themysubsubsec}#1}}
\setcounter{secdepth}{3}

%--------------------------------------------- Tabellen

\usepackage{array}
\usepackage{tabularx}
\usepackage{siunitx}
\usepackage{booktabs}
\usepackage{multirow}
\usepackage{multicol}
\newcommand{\tabH}[1]{\parbox[1pt][#1em][c]{0cm}{}}
\newcommand{\red}[1]{\textcolor{red}{#1}}
\newcommand{\blue}[1]{\textcolor{blue}{#1}}
\newcommand{\orange}[1]{\textcolor{orange}{#1}}

\newcommand{\schaum}[1]{\textcolor{red!50}{\scriptsize Schaum S.#1}}
\newcommand{\buch}[1]{\textcolor{red!70!white}{\scriptsize Buch Seite #1}}
\newcommand{\skript}[1]{\textcolor{red!70!white}{\scriptsize Skript Seite #1}}
\newcommand{\uZ}{\underline{Z}}

%-------------- Eigene Befehle, nur tabularx Umgebung
\newcolumntype{Y}{>{\small\raggedright\arraybackslash}X}
\newcolumntype{m}{>{\parbox[1pt][2em][c]{0cm}{}\centering\arraybackslash}X}
\newcolumntype{n}{>{\parbox[1pt][2.5em][c]{0cm}{}\centering\arraybackslash}X}
\newcolumntype{M}{>{\parbox[1pt][3em][c]{0cm}{}\centering\arraybackslash}X}
\newcolumntype{B}{>{\parbox[1pt][4em][c]{0cm}{}\centering\arraybackslash}X}

%--------------------------------------------- Mathe packages

\usepackage{relsize}					% f�r grosse Summen- und Integralzeichen \mathlarger\int\limits_{min}^{max}
\usepackage{amsmath}
\DeclareMathOperator{\arccot}{arccot}
\DeclareMathOperator{\arsinh}{arsinh}
\DeclareMathOperator{\arcosh}{arcosh}
\DeclareMathOperator{\artanh}{artanh}
\DeclareMathOperator{\arcoth}{arcoth}
\DeclareMathOperator{\cjs}{cjs}
\DeclareMathOperator{\IM}{Im}
\DeclareMathOperator{\RE}{Re}
\DeclareMathOperator{\Ln}{Ln}
\DeclareMathOperator{\sgn}{sgn}
\DeclareMathOperator{\Var}{Var}
\DeclareMathOperator{\var}{var}
\DeclareMathOperator{\cov}{cov}
\DeclareMathOperator{\sinc}{sinc}

%\newcommand{\korrpaar}{\hspace{1em}\circ -\hspace{0.8mm} \bullet\hspace{1em}}
\unitlength1cm
% Zeitbereich -- Frequenzbereich
\newcommand{\korrpaar}
{
\begin{picture}(1,0.5)
\put(0.2,0.1){\circle{0.14}}\put(0.27,0.1){\line(1,0){0.5}}\put(0.77,0.1){\circle*{0.14}}
\end{picture}
}
% Frequenzbereich -- Zeitbereich
\newcommand{\Korrpaar}
{
\begin{picture}(1,0.5)
\put(0.2,0.1){\circle*{0.14}}\put(0.27,0.1){\line(1,0){0.45}}\put(0.80,0.1){\circle{0.14}}
\end{picture}
}
%--------------------------------------------- Nummerierung von Gleichungen

%Setzt den equation-Zaehler nach jeder Seite zurueck
\numberwithin{equation}{section}	

%Definiert den Stil:
\renewcommand{\theequation}{\arabic{section}.\arabic{equation}}

%--------------------------------------------- Graue Beispielboxen

\definecolor{MyBoxColor}{rgb}{0.9,0.9,0.9}
\newenvironment{shadedSmaller}{
  \def\FrameCommand{\fboxsep=\FrameSep \colorbox{MyBoxColor}}
  \MakeFramed {\advance\hsize-2\width\FrameRestore}}
{\endMakeFramed}

\newenvironment{Beispiel}{
  \def\FrameCommand{\fboxsep=1em \colorbox{MyBoxColor}}
  \MakeFramed {\advance\hsize-1.1\width\FrameRestore}}
{\endMakeFramed}

\newcommand{\loesung}{\subsubsection*{L�sung}}

\newenvironment{teilaufgaben}{
\begin{enumerate}
\renewcommand{\labelenumi}{\alph{enumi})}
}{\end{enumerate}}

%---------------------------------------------

%------------------- C-Code ------------------- %
\definecolor{colorkeywordstyle}{RGB}{127,0,127}
\definecolor{colorcommentstyle}{RGB}{0,116,0}
\definecolor{colorstringstyle}{RGB}{196,26,22}
\definecolor{colorbackground}{RGB}{240,240,240}
\definecolor{royalblue}{rgb}{0.15,0.25,0.55}
\definecolor{black}{rgb}{0,0,0}

\usepackage{listings}
\lstnewenvironment{code}[1][]
  {\lstset{
  	backgroundcolor =		\color{colorbackground},
    language =				C++,			
    basicstyle =    		\ttfamily\small,
    breaklines = 			true,
    frame = 				none,
    columns =				flexible,
    keepspaces = 			true,
    keywordstyle=			\color{colorkeywordstyle},
    identifierstyle	=\color{royalblue},
    showstringspaces =		false,
    extendedchars =			true,   
    commentstyle = 			\color{colorcommentstyle},,
    numbers = 				left,
    numbersep = 			2pt,
    numberstyle =			\tiny,
    rulecolor=\color{black}, 
    stepnumber =			1,
    tabsize = 				4,
  }}{\vspace{0em}}
%---------------------------------------------

\newcommand{\dunderline}[1]{\underline{\underline{#1}}}

\begin{document}

\tableofcontents

\section{Aufgabenstellung}
Um ein Problem der Mathematik besser zu verstehen, ist es oft sehr hilfreich, wenn dieses verst�ndlich beschrieben ist. Eine Visualisierung in Form eines Bildes oder kurzen Films ist dabei eine M�glichkeit dies zu bewerkstelligen.

Die Hauptaufgabe dieses Projektes, war es nun, die Greensche Funktion in zwei Dimensionen zu visualisieren. Die L�sung f�r eine Dimension wurde bereits in einem kurzen Film veranschaulicht \footnote{\url{https://www.youtube.com/watch?v=Wpi7Gf7V2HY}}. Es soll dabei zuerst eine partielle Differentialgleichung mithilfe des Greenschen Ansatzes gel�st und danach das visualisieren umgesetzt werden. Die Greensche Funktion erm�glicht die Berechnung eines Problems durch Supperposition. Darum sollten auch verschiedene Anfangswerte untersucht werden und wie diese sich auf die L�sungen auswirken. 

Als Differentialgleichung sei dabei ein Potentialproblem vorgegeben. Vorliegend ist eine leitende Platte, die am Rande geerdet sei. Wenn nun ein Potential an einem oder mehreren Punkten auf dieser Platte anliegt, ist es von Interesse zu wissen, welches Potential man nun an einem beliebigen Punkt auf der Platte misst. Die folgende partielle Differentialgleichung l�st dabei dieses Problem.
\begin{equation}
	\dfrac{\partial^2 u}{\partial x^2}+\dfrac{\partial^2 u}{\partial y^2} = f(x,y).
\end{equation}
Dabei sei hier noch auf die Arbeit von Reto Christen und Philip Solenthaler verwiesen. Diese haben das Potentialproblem genauer untersucht. 

\section{Algorithmus}

	Da wir aus einer Matrix von Werten eine andere, gleich grosse  Matrix mit weiteren Werten berechnen, m�ssen wir zuerst diese Matrix in einen Vektor zerlegen.
	
	\begin{equation}
		\dunderline{A}\cdot \dunderline{x} \ne \dunderline{f} \qquad\Rightarrow\qquad \dunderline{A}\cdot \underline{x} = \underline{f}
		\label{eq:gleichung}
	\end{equation}
	
	Dabei haben wir bei beiden Matrizen die Spalten untereinander gereiht und so je einen Vektor mit der L�nge $n^2$ erhalten. Mathematisch macht es keinen Unterschied, solange wir am Schluss den L�sungsvektor wieder auf die selbe Art und Weise zu einer Matrix zusammenf�gen.
	
\[
	A=\left(
	\begin{array}{ccccc|ccccc|c|ccccc}
	    -4&     1&     0&\cdots&     0 &     1&     0&     0&\cdots&     0 &\cdots &      &      &      &      &      \\
	     1&    -4&     1&\cdots&     0 &     0&     1&     0&\cdots&     0 &\cdots &      &      &      &      &      \\
	     0&     1&    -4&\cdots&     0 &     0&     0&     1&\cdots&     0 &\cdots &      &      &     0&      &      \\
	\vdots&\vdots&\vdots&\ddots&\vdots &\vdots&\vdots&\vdots&\ddots&\vdots &       &      &      &      &      &      \\
	     0&     0&     0&\cdots&    -4 &     0&     0&     0&\dots &     1 &\cdots &      &      &      &      &      \\
	\hline
	     1&     0&     0&\cdots&     0 &    -4&     1&     0&\dots &     0 &\cdots &      &      &      &      &      \\
	     0&     1&     0&\cdots&     0 &     1&    -4&     1&\dots &     0 &\cdots &      &      &      &      &      \\
	     0&     0&     1&\cdots&     0 &     0&     1&    -4&\dots &     0 &\cdots &      &      &     0&      &      \\
	\vdots&\vdots&\vdots&\ddots&\vdots &\vdots&\vdots&\vdots&\ddots&\vdots &       &      &      &      &      &      \\
	     0&     0&     0&\cdots&     1 &     0&     0&     0&\cdots&    -4 &\cdots &      &      &      &      &      \\
	\hline
	\vdots&\vdots&\vdots&      &\vdots &\vdots&\vdots&\vdots&      &\vdots &\ddots &\vdots&\vdots&\vdots&      &\vdots\\
	\hline
	      &      &      &      &       &      &      &      &      &       &\cdots &    -4&     1&     0&\cdots&     0\\
	      &      &      &      &       &      &      &      &      &       &\cdots &     1&    -4&     1&\cdots&     0\\
	      &      &     0&      &       &      &      &     0&      &       &\cdots &     0&     1&    -4&\cdots&     0\\
	      &      &      &      &       &      &      &      &      &       &       &\vdots&\vdots&\vdots&\ddots&\vdots\\
	      &      &      &      &       &      &      &      &      &       &\cdots &     0&     0&     0&\cdots&    -4\\
	\end{array}
	\right) 
	\]
	
	Die Matrix $A$ enth�lt die Koeffizienten f�r die partielle Differentialgleichung der zweiten Ordnung. Sie hat die Gr�sse $n^2 \times n^2$. Diese Matrix auf dem Heap zu allozieren w�re f�r mittelgrosse Matrizen schon nicht mehr m�glich.\cite{mueller:hpcseminar}
	
	
	
	Der ben�tigte Speicher f�r eine Matrix $f$ mit der Gr�sse $500 \times 500$ und \verb|float| Werten (4 Byte) w�re
	
	\begin{equation}
		500^2 \times 500^2 \cdot 4\;\mathrm{Byte} = 232.8\;\mathrm{GByte}
	\end{equation}
	
	Mit $n = 1000$ sogar 3.64\;TByte, also definitiv zu viel. Dies ist aber auch gar nicht n�tig. Es m�ssen jeweils nur die Nachbarelemente von $f_k$ addiert und durch einen konstanten Faktor geteilt werden, die Randelemente sind dabei Null zu setzen.

	\begin{eqnarray}
		f_k = U_{k-1}+U_{k+1}+U_{k-(n-1)}+U_{k+(n-1)}-4U_k
	\end{eqnarray}
	
	Da die diskrete partielle Ableitung jeweils nur die Elemente ober/unterhalb und links/recht des aktuellen Elementes in die Rechnung mit einbezieht, m�ssen mindestens $n$ Iterationen durchgef�hrt werden, damit sich das Potential �ber die ganze Ebene verteilt.
	
	\subsection{Parallelisierung}
		
		Um das Gleichungssystem (\ref{eq:gleichung}) zu l�sen, haben wir den Gauss-Seidel Algorithmus benutzt. Bei diesem Algorithmus ist, wie wir wissen, die aktuelle Zeile von der vorherigen abh�ngig. Bei der Parallelisierung wird das Gleichungssystem an verschiedenen Stellen zu l�sen begonnen. Das ist zwar nicht so effizient, wie ein "'normaler"' Gauss-Seidel, wird aber durch die Parallelisierung schneller.
		
		Die Abweichung ist bei den ersten Schritten am gr�ssten, und wird bei jeder Iteration kleiner. Als wie die ersten Berechnungen durch gef�hrt hatten, fiel uns auf, dass die einzelnen Threads am Anfang gut als Linien von Spitzen sichtbar sind (\fref{fig:201_1}).
		
		\begin{figure}[h]
			\centering
			\includegraphics[width = 15cm]{./images/step001}
			\caption{Berechnung von einer $201 \times 201$ Fl�che mit 32 Threads nach dem ersten Iterationsschritt}
			\label{fig:201_1}
		\end{figure}
		
		
		Wir haben uns f�r OpenMP entschieden, da wir ein Grosses Problem haben, welches immer auf den selben Speicher zugreift. Wie schon beim Kugelsternhaufen erw�hnt, ist die Parallelisierung einfach zu realisieren:
		
\begin{code}
	int numthreads = 32;
	#pragma omp parallel for num_threads(numthreads)
		for (i = 0; i < dim; i++)
		{ ...
\end{code}
		
		
		
		


\section{Visualisierung}
	
	Wir speichern die erhaltene Matrix in einem .csv File mit einer Genauigkeit von vier Nachkommastellen ab, was absolut hinreichen f�r eine Visualisierung ist. Anfangs benutzten wir MATLAB um aus dem CSV-File eine anst�ndige Visualisierung zu erhalten. Der Aufwand war gering und die Resultate waren schnell kontrollierbar.
	
	Um das zu automatisieren benutzen wir sp�ter Gnuplot, welches sich mit C-Code ohne Probleme ansprechen l�sst und uns die Bilder automatisch mit dem vorgegebenen Einstellungen erstellt.
	
	Gnuplot ist ein Kommandozeilen basiertes OpenSource Tool, das aber auch mit GUI erh�ltlich ist. �ber eine Pipe lassen sich lassen sich alle Einstellungen vornehmen und aus dem .CSV-File ein PNG generieren.
	
	Mit der Green-Funtion k�nnen bei einer partiellen DGL 2. Ordnung einzelne Punkte berechnet werden. Um das zu visualisieren, berechnen wir einzelne Ringe bzw. Quadrate vom Zentrum ausgehen nach aussen. Man k�nnte nat�rlich auch jeder Punkt einzeln dazu nehmen, was mehr Berechnungen erfordert.
	

	
	Diese einzelnen Schritte werden vollst�ndig berechnet und dann als .csv abgespeichert. Die Bilder werden nachdem alle Einzelschritte berechnet wurden aus den Dateien wiederum parallelisiert in einer \verb|for|-Schleife zu .png Bilddateien verarbeitet. Die Skalierung der z-Achse wird direkt aus den Daten des letzten Schrittes berechnet, was uns eine optimale Darstellung garantiert. Aus den Bilddateien wird am Schluss optional noch ein Video erstellt.
		


\end{document}