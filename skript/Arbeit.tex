\documentclass[a4paper, parskip=half, DIV15]{scrartcl}
\usepackage[T1]{fontenc}
\usepackage[latin1]{inputenc}
\usepackage[ngerman]{babel}
%-------------- Kopf- und Fusszeile

\newcommand{\authorinfo}{Andreas Linggi, Stefan Steiner}
% �usserer Kopf
		\newcommand{\oheaderinfo}{Mathematisches Seminar}
% innerer Kopf
		\newcommand{\iheaderinfo}{}
% zentrierter Kopf
		\newcommand{\cheaderinfo}{}
%------- Rand falls eingeschaltet
%\newcommand{\randabstand}{1.5cm}
%--------------------------------------------- Schriften
%------- Schriften Kopf- und Fusszeile
\newcommand{\headfootnotefont}{\scriptsize\upshape}
%------- Schrift von Listenelementen
\newcommand{\listfont}{\sffamily}
%--------------------------------------------- Name von Eigener Liste
\newcommand{\nameoflist}{\oheaderinfo \hspace{1cm}\small{\authorinfo} \hspace{3cm}\today}

%\usepackage[style=ieee,backend=bibtex]{biblatex}
\usepackage[style = ieee,backend=bibtex]{biblatex}
\bibliography{quellen}

%\usepackage[	left	=1.5cm, %
%				right	=1cm, %
%				top		=1cm, %
%				bottom	=1cm, %
%includeheadfoot]{geometry}

\usepackage[ngerman]{varioref}
\usepackage[german]{fancyref}
\usepackage{graphicx,amssymb,amsmath}
\usepackage{framed}						% F�r Graue Beispielboxen
\usepackage{epstopdf}					% Zum einbinden von .eps Dateien (z.B. Matlab)
\usepackage{stmaryrd}					% Dreieck-Sternschaltung
%--------------------------------------------- Dsrstellung der Einheiten
\usepackage[locale=DE]{siunitx}
\sisetup{per-mode=symbol}
%--------------------------------------------- Kopf- und Fusszeile

\usepackage{scrpage2}
\pagestyle{scrheadings}
\clearscrheadings
\clearscrplain
\ifoot{\headfootnotefont\authorinfo}
%------ % SeitenZahlen outer foot
\usepackage{lastpage}
\ofoot{\headfootnotefont Seite \thepage { }von \pageref{LastPage}}
%------
\ohead{\headfootnotefont\oheaderinfo}
\ihead{\headfootnotefont\iheaderinfo}
\chead{\headfootnotefont\cheaderinfo}
\setheadsepline{.4pt} 											% Linie unter der Kopfzeile

%--------------------------------------------- TikZ und GNU Plot
\usepackage{tikz}
\usetikzlibrary{arrows,shapes,trees,calc,patterns}
\usepgflibrary{arrows}
\usepackage{pgfplots}
\usepackage[europeanresistors,europeaninductors]{circuitikz} % F�r Schaltelemente

\pgfdeclarelayer{bg}    % declare background layer
\pgfsetlayers{bg,main}
\newcommand*{\rechterWinkelRadius}{.25cm}
\newcommand*{\rechterWinkel}[2]{% #1 = point, #2 = start angle
   \draw[shift={(#2:\rechterWinkelRadius)}] (#1) arc[start angle=#2, delta angle=90, radius = \rechterWinkelRadius];
   \fill[shift={(#2+45:\rechterWinkelRadius/2)}] (#1) circle[radius=1.25\pgflinewidth];
}


%---------------------------------------------




%--------------------------------------------- & Eigene Liste mysec
\usepackage{hyperref}
\usepackage{tocloft}

%\newlistof{mytopsec}{sec}{\nameoflist}
%\newcommand{\mytopsec}[1]{%
%\refstepcounter{mytopsec}
%\par\noindent{\textsf{\textbf{\themytopsec.\hspace{0.5em} #1}}}
%\addcontentsline{sec}{mytopsec}{\protect\numberline{\themytopsec}#1}\par}
%
%

\newlistof{mysec}{sec}{\nameoflist}
\newcommand{\mysec}[1]{%
\refstepcounter{mysec}
\par\noindent{\textsf{\textbf{\themysec.\hspace{0.5em} #1}}}
\addcontentsline{sec}{mysec}{\protect\numberline{\themysec}#1}\par}
\newlistentry[mysec]{mysubsec}{sec}{1}
\newcommand{\mysubsec}[1]{%
\refstepcounter{mysubsec}
\par\noindent{\textsf{\themysubsec\hspace{0.5em}#1}}
\addcontentsline{sec}{mysubsec}{\protect\numberline{\themysubsec}#1}}
\setcounter{secdepth}{2}
\newlistentry[mysubsec]{mysubsubsec}{sec}{2}
\newcommand{\mysubsubsec}[1]{%
\refstepcounter{mysubsubsec}
\par\noindent{\textsf{\themysubsubsec\hspace{0.5em}#1}}
\addcontentsline{sec}{mysubsubsec}{\protect\numberline{\themysubsubsec}#1}}
\setcounter{secdepth}{3}

%--------------------------------------------- Tabellen

\usepackage{array}
\usepackage{tabularx}
\usepackage{siunitx}
\usepackage{booktabs}
\usepackage{multirow}
\usepackage{multicol}
\newcommand{\tabH}[1]{\parbox[1pt][#1em][c]{0cm}{}}
\newcommand{\red}[1]{\textcolor{red}{#1}}
\newcommand{\blue}[1]{\textcolor{blue}{#1}}
\newcommand{\orange}[1]{\textcolor{orange}{#1}}

\newcommand{\schaum}[1]{\textcolor{red!50}{\scriptsize Schaum S.#1}}
\newcommand{\buch}[1]{\textcolor{red!70!white}{\scriptsize Buch Seite #1}}
\newcommand{\skript}[1]{\textcolor{red!70!white}{\scriptsize Skript Seite #1}}
\newcommand{\uZ}{\underline{Z}}

%-------------- Eigene Befehle, nur tabularx Umgebung
\newcolumntype{Y}{>{\small\raggedright\arraybackslash}X}
\newcolumntype{m}{>{\parbox[1pt][2em][c]{0cm}{}\centering\arraybackslash}X}
\newcolumntype{n}{>{\parbox[1pt][2.5em][c]{0cm}{}\centering\arraybackslash}X}
\newcolumntype{M}{>{\parbox[1pt][3em][c]{0cm}{}\centering\arraybackslash}X}
\newcolumntype{B}{>{\parbox[1pt][4em][c]{0cm}{}\centering\arraybackslash}X}

%--------------------------------------------- Mathe packages

\usepackage{relsize}					% f�r grosse Summen- und Integralzeichen \mathlarger\int\limits_{min}^{max}
\usepackage{amsmath}
\DeclareMathOperator{\arccot}{arccot}
\DeclareMathOperator{\arsinh}{arsinh}
\DeclareMathOperator{\arcosh}{arcosh}
\DeclareMathOperator{\artanh}{artanh}
\DeclareMathOperator{\arcoth}{arcoth}
\DeclareMathOperator{\cjs}{cjs}
\DeclareMathOperator{\IM}{Im}
\DeclareMathOperator{\RE}{Re}
\DeclareMathOperator{\Ln}{Ln}
\DeclareMathOperator{\sgn}{sgn}
\DeclareMathOperator{\Var}{Var}
\DeclareMathOperator{\var}{var}
\DeclareMathOperator{\cov}{cov}
\DeclareMathOperator{\sinc}{sinc}

%\newcommand{\korrpaar}{\hspace{1em}\circ -\hspace{0.8mm} \bullet\hspace{1em}}
\unitlength1cm
% Zeitbereich -- Frequenzbereich
\newcommand{\korrpaar}
{
\begin{picture}(1,0.5)
\put(0.2,0.1){\circle{0.14}}\put(0.27,0.1){\line(1,0){0.5}}\put(0.77,0.1){\circle*{0.14}}
\end{picture}
}
% Frequenzbereich -- Zeitbereich
\newcommand{\Korrpaar}
{
\begin{picture}(1,0.5)
\put(0.2,0.1){\circle*{0.14}}\put(0.27,0.1){\line(1,0){0.45}}\put(0.80,0.1){\circle{0.14}}
\end{picture}
}
%--------------------------------------------- Nummerierung von Gleichungen

%Setzt den equation-Zaehler nach jeder Seite zurueck
\numberwithin{equation}{section}	

%Definiert den Stil:
\renewcommand{\theequation}{\arabic{section}.\arabic{equation}}

%--------------------------------------------- Graue Beispielboxen

\definecolor{MyBoxColor}{rgb}{0.9,0.9,0.9}
\newenvironment{shadedSmaller}{
  \def\FrameCommand{\fboxsep=\FrameSep \colorbox{MyBoxColor}}
  \MakeFramed {\advance\hsize-2\width\FrameRestore}}
{\endMakeFramed}

\newenvironment{Beispiel}{
  \def\FrameCommand{\fboxsep=1em \colorbox{MyBoxColor}}
  \MakeFramed {\advance\hsize-1.1\width\FrameRestore}}
{\endMakeFramed}

\newcommand{\loesung}{\subsubsection*{L�sung}}

\newenvironment{teilaufgaben}{
\begin{enumerate}
\renewcommand{\labelenumi}{\alph{enumi})}
}{\end{enumerate}}

%---------------------------------------------

\begin{document}
\title{Visualisierung der Greenschen Funktion}
\maketitle
\tableofcontents

\section{Aufgabenstellung}
Um ein Problem der Mathematik besser zu verstehen, ist es oft sehr hilfreich, wenn dieses verst�ndlich beschrieben ist. Eine Visualisierung in Form eines Bildes oder kurzen Films ist dabei eine M�glichkeit dies zu bewerkstelligen.

Die Hauptaufgabe dieses Projektes, war es nun, die Greensche Funktion in zwei Dimensionen zu visualisieren. Die L�sung f�r eine Dimension wurde bereits in einem kurzen Film veranschaulicht \footnote{\url{https://www.youtube.com/watch?v=Wpi7Gf7V2HY}}. Es soll dabei zuerst eine partielle Differentialgleichung mithilfe des Greenschen Ansatzes gel�st und danach das visualisieren umgesetzt werden. Die Greensche Funktion erm�glicht die Berechnung eines Problems durch Supperposition. Darum sollten auch verschiedene Anfangswerte untersucht werden und wie diese sich auf die L�sungen auswirken. 

Als Differentialgleichung sei dabei ein Potentialproblem vorgegeben. Vorliegend ist eine leitende Platte, die am Rande geerdet sei. Wenn nun ein Potential an einem oder mehreren Punkten auf dieser Platte anliegt, ist es von Interesse zu wissen, welches Potential man nun an einem beliebigen Punkt auf der Platte misst. Die folgende partielle Differentialgleichung l�st dabei dieses Problem.
\begin{equation}
	\dfrac{\partial^2 u}{\partial x^2}+\dfrac{\partial^2 u}{\partial y^2} = f(x,y).
\end{equation}
Dabei sei hier noch auf die Arbeit von Reto Christen und Philip Solenthaler verwiesen. Diese haben das Potentialproblem genauer untersucht. 

%------------------- C-Code ------------------- %
\definecolor{colorkeywordstyle}{RGB}{127,0,127}
\definecolor{colorcommentstyle}{RGB}{0,116,0}
\definecolor{colorstringstyle}{RGB}{196,26,22}
\definecolor{colorbackground}{RGB}{240,240,240}
\definecolor{royalblue}{rgb}{0.15,0.25,0.55}
\definecolor{black}{rgb}{0,0,0}

\lstnewenvironment{code}[1][]
  {\lstset{
  	backgroundcolor =		\color{colorbackground},
    language =				C++,			
    basicstyle =    		\ttfamily,
    breaklines = 			true,
    frame = 				none,
    columns =				flexible,
    keepspaces = 			true,
    keywordstyle=			\color{colorkeywordstyle},
    identifierstyle	=\color{royalblue},
    showstringspaces =		false,
    extendedchars =			true,   
    commentstyle = 			\color{colorcommentstyle},,
    numbers = 				left,
    numbersep = 			2pt,
    numberstyle =			\tiny,
    rulecolor=\color{black}, 
    stepnumber =			1,
    tabsize = 				4,
  }}{\vspace{0em}}
\lstset{
  	backgroundcolor =		\color{colorbackground},
    language =				C++,			
    basicstyle =    		\ttfamily,
    breaklines = 			true,
    frame = 				none,
    columns =				flexible,
    keepspaces = 			true,
    keywordstyle=			\color{colorkeywordstyle},
    identifierstyle	=\color{royalblue},
    showstringspaces =		false,
    extendedchars =			true,   
    commentstyle = 			\color{colorcommentstyle},,
    numbers = 				left,
    numbersep = 			2pt,
    numberstyle =			\tiny,
    rulecolor=\color{black}, 
    stepnumber =			1,
    tabsize = 				4,
}
%---------------------------------------------
\newcommand{\dunderline}[1]{\underline{\underline{#1}}}

%------------------- Graue Beispielboxen ------------------- %
\definecolor{MyBoxColor}{rgb}{0.9,0.9,0.9}
\newenvironment{shadedSmaller}{
  \def\FrameCommand{\fboxsep=\FrameSep \colorbox{MyBoxColor}}
  \MakeFramed {\advance\hsize-2\width\FrameRestore}}
{\endMakeFramed}
%----------------------------------------------------------- %

\section{Aufgabenstellung}

	Um ein Problem der Mathematik besser zu verstehen, ist es oft sehr hilfreich, wenn es verst�ndlich beschrieben ist. Eine Visualisierung in Form eines Bildes oder kurzen Films ist dabei eine M�glichkeit dies zu bewerkstelligen.
	
	Ziel dieses Projektes war, die Greensche Funktion in zwei Dimensionen zu visualisieren. Die L�sung f�r eine Dimension wurde bereits in einem kurzen Film veranschaulicht \footnote{\url{https://www.youtube.com/watch?v=Wpi7Gf7V2HY}}. Es soll dabei zuerst eine partielle Differentialgleichung gel�st und danach visualisiert werden. Die Greensche Funktion erm�glicht die Berechnung eines Problems durch Superposition. Darum sollten auch verschiedene Anfangswerte untersucht werden und wie diese sich auf die L�sungen auswirken. 
	
	Als Differentialgleichung ist das Potentialproblem vorgegeben. Gegeben ist eine leitende Platte, die am Rande geerdet sei. Wenn nun ein Potential an einem oder mehreren Punkten auf dieser Platte anliegt, ist es von Interesse zu wissen, welches Potential man nun an einem beliebigen Punkt auf der Platte misst. 

\section{George Green}

	Der Namensgeber f�r die Greensche Funktion ist George Green. Er war ein britischer Physiker und Mathematiker. Geboren wurde er 1793 in der N�he von Nottingham. Sein Vater betrieb eine M�hle und George Green arbeitete ebenfalls als M�ller. Nach dem Tod seines Vaters f�hrte er den M�hlenbetrieb fort. Bemerkenswert ist, dass Green nur etwa zwei Jahre in die Schule ging. Mathematische und Physikalische Grundlagen brachte er sich selber bei. Der Ort an dem er lernte war seine M�hle. Da nie ein Portrait von ihm angefertigt wurde, gibt es kein Bild von ihm. Darum wird anstatt seinem Konterfei jeweils eine Windm�hle verwendet um ihn darzustellen. Die Windm�hle gibt es �brigens immer noch. Green ver�ffentliche mit etwa dreissig Jahren seine erste Arbeit. Diese wurde kaum beachtet, ausser von einem adligen Mathematik namens Sir Edward Bromhead. Er ermutigte Green, im Alter von vierzig Jahren in Cambridge zu studieren. Interessant zu wissen ist dazu, dass dort zu dieser Zeit die Theorien von Laplace und Fourier noch nicht gelehrt wurden. Green hatte sich jedoch durch diese Theorie gelesen und sogar noch einige Erweiterungen hinzugef�gt. Vier Jahre nachdem er graduierte und kurz vor seinem internationalen Durchbruch stand, starb er 1841 jedoch an einer schweren Grippe. Dadurch geriet seine Arbeit f�r einige Zeit in Vergessenheit.
	
	George Green ist besonders bekannt f�r das Greensche Theorem und die Greensche Funktion. Er besch�ftigte sich mit der L�sung von partiellen Differentialgleichungen und stellte damit unter anderem mathematische Hilfsmittel bereit, die sogar in der Quantenfeldtheorie im 20. Jahrhundert von Bedeutung waren.\cite{wiki:green}
	
	Neben vielen anderen Pers�nlichkeiten wie Maxwell, Dirac oder Newton ist er in der Westminster Abbey in London mit einer Gedenktafel verewigt\cite{wiki:westminster}. Der Grund, warum George Green nicht so popul�r ist wie Gauss oder Maxwell, liegt wahrscheinlich darin, dass von seiner Zeit, als er in seiner M�hle gearbeitet hat, wenig  bekannt ist. Auch war er nur f�r sehr kurze Zeit in Cambridge t�tig. 

	\begin{figure}                    
	\centering 
	\includegraphics[width=8cm]{images/greens_windmill} 
	\caption{Windm�hle, in der Green f�r lange Zeit gearbeitet hat} 
	\label{fig:abb1} 
	\end{figure} 
	
\section{Algorithmus}
	Um eine Visualisierung �berhaupt durchf�hren zu k�nnen, braucht es in unserem Fall Daten, mit denen wir arbeiten k�nnen. Diese liefert uns ein Programm, bei welchem wir Daten einspeisen und im Gegenzug bearbeitet erhalten. 
	
	Das Problem, welches wir hier l�sen, wurde ebenfalls von Philip Solenthaler und Reto Christen bearbeitet. Sie haben untersucht wie ein Programm geschrieben werden kann, welches dieses Problem m�glichst effizient l�st.
	\subsection{Mathematischer Ansatz zur L�sung}

	Das Potentialproblem auf der leitenden Platte, welches wir l�sen m�chten, hat nun folgende partielle Differentialgleichung. 
	\begin{equation}\label{eq:gleichung}
		\dfrac{\partial^2 u}{\partial x^2}+\dfrac{\partial^2 u}{\partial y^2} = f(x,y).
	\end{equation}
	Wobei die Rand-, und Anfangsbedingungen auf der leitenden Platte alle null sind. 
	
	Die quadratische Platte mit der L�nge $n\times n$ wird dabei mit den Werten $u_{ij}$ diskretisiert, wobei $0 < i,j<n$. Wir bekommen eine endliche Anzahl an Werte $u$.
%	 Dabei wird f�r jeden Wert $u$ diese Gleichung gel�st. 
	Der Ansatz f�r eine L�sung dieses Problems bietet die lineare Algebra, welcher bereits im Theorieteil des Skriptes ausf�hrlich hergeleitet wird. Das bedeutet, wir m�ssen ein Problem folgender Art l�sen.
	\begin{equation}
		Au = f
	\end{equation}
	Es sei dabei $A$ die Koeffizientenmatrix, $f$ der St�rvektor und $u$ der L�sungsvektor. 
	
	Die leitende Platte wird als eine Fl�che visualisiert. Um diese als Vektor $u$ zu schreiben ist dabei festgelegt worden, dass die Spalten dieser diskretisierten Fl�che untereinander geschrieben werden. Dies ergibt Vektoren der l�nge $n^2$.
	
	Aus diesem Zusammenhang heraus ergibt sich dabei die Koeffizientenmatrix $A$. 
\[
	A=\left(
	\begin{array}{ccccc|ccccc|c|ccccc}
	    -4&     1&     0&\cdots&     0 &     1&     0&     0&\cdots&     0 &\cdots &      &      &      &      &      \\
	     1&    -4&     1&\cdots&     0 &     0&     1&     0&\cdots&     0 &\cdots &      &      &      &      &      \\
	     0&     1&    -4&\cdots&     0 &     0&     0&     1&\cdots&     0 &\cdots &      &      &     0&      &      \\
	\vdots&\vdots&\vdots&\ddots&\vdots &\vdots&\vdots&\vdots&\ddots&\vdots &       &      &      &      &      &      \\
	     0&     0&     0&\cdots&    -4 &     0&     0&     0&\dots &     1 &\cdots &      &      &      &      &      \\
	\hline
	     1&     0&     0&\cdots&     0 &    -4&     1&     0&\dots &     0 &\cdots &      &      &      &      &      \\
	     0&     1&     0&\cdots&     0 &     1&    -4&     1&\dots &     0 &\cdots &      &      &      &      &      \\
	     0&     0&     1&\cdots&     0 &     0&     1&    -4&\dots &     0 &\cdots &      &      &     0&      &      \\
	\vdots&\vdots&\vdots&\ddots&\vdots &\vdots&\vdots&\vdots&\ddots&\vdots &       &      &      &      &      &      \\
	     0&     0&     0&\cdots&     1 &     0&     0&     0&\cdots&    -4 &\cdots &      &      &      &      &      \\
	\hline
	\vdots&\vdots&\vdots&      &\vdots &\vdots&\vdots&\vdots&      &\vdots &\ddots &\vdots&\vdots&\vdots&      &\vdots\\
	\hline
	      &      &      &      &       &      &      &      &      &       &\cdots &    -4&     1&     0&\cdots&     0\\
	      &      &      &      &       &      &      &      &      &       &\cdots &     1&    -4&     1&\cdots&     0\\
	      &      &     0&      &       &      &      &     0&      &       &\cdots &     0&     1&    -4&\cdots&     0\\
	      &      &      &      &       &      &      &      &      &       &       &\vdots&\vdots&\vdots&\ddots&\vdots\\
	      &      &      &      &       &      &      &      &      &       &\cdots &     0&     0&     0&\cdots&    -4\\
	\end{array}
	\right) 
	\]

\subsection{Technische Umsetzung}

	Die Matrix $A$ enth�lt die Koeffizienten f�r die partielle Differentialgleichung der zweiten Ordnung. Sie hat die Gr�sse $n^2 \times n^2$. Diese Matrix im Arbeitsspeicher abzuspeichern w�re f�r mittelgrosse Matrizen schon nicht mehr m�glich.\cite{mueller:hpcseminar}
	
	Der ben�tigte Speicher f�r eine Platte mit der Gr�sse $500 \times 500$ und \verb|float| Werten (4 Byte) w�re
	
	\begin{equation}
		500^2 \times 500^2 \cdot 4\;\mathrm{Byte} = 232.8\;\mathrm{GByte}
	\end{equation}
	
	Mit $n = 1000$ sogar 3.64\;TByte, also definitiv zu viel. Dies ist aber auch gar nicht n�tig. Es m�ssen jeweils nur die Nachbarelemente von $u_{ij}$ addiert und durch einen konstanten Faktor geteilt werden, die Randelemente sind dabei Null zu setzen.

	\begin{eqnarray}
%		f_{ij} = u_{i-i,j}+u_{i+1,j}+u_{i,j+1}+u_{i,j-1}-4u_{ij}\\
		u_{ij} = \dfrac{u_{i-i,j}+u_{i+1,j}+u_{i,j+1}+u_{i,j-1}-f_{ij}}{4} \label{eq:gauss-seidel}
	\end{eqnarray}
	
	Da die diskretisierte partielle Ableitung jeweils nur die Elemente ober/unterhalb und links/recht des aktuellen Elementes in die Rechnung mit einbezieht (\ref{eq:gauss-seidel}), m�ssen mindestens $n$ Iterationen durchgef�hrt werden, damit sich das Potential �ber die ganze Ebene verteilt.
	
	\subsubsection{Parallelisierung}
	
	Um das Gleichungssystem (\ref{eq:gleichung}) zu l�sen, haben wir den Gauss-Seidel Algorithmus benutzt. Bei diesem Algorithmus ist die aktuelle Zeile von der vorherigen abh�ngig. Bei der Parallelisierung wird das Gleichungssystem an verschiedenen Stellen zu l�sen begonnen. Das ist zwar nicht so effizient, wie ein "'normaler"' Gauss-Seidel, wird aber durch die Parallelisierung schneller.
		
		Die Abweichung ist bei den ersten Schritten am gr�ssten, und wird bei jeder Iteration kleiner. Als wie die ersten Berechnungen durch gef�hrt hatten, fiel uns auf, dass die einzelnen Threads am Anfang gut als Linien von Spitzen sichtbar sind (\fref{fig:201_1}).
		
		\begin{figure}
			\centering
			\includegraphics[width = 15cm]{./images/step001}
			\caption{Berechnung von einer $201 \times 201$ Fl�che mit 32 Threads nach dem ersten Iterationsschritt}
			\label{fig:201_1}
		\end{figure}
		
		
		Wir haben uns f�r Open-MP entschieden, da es f�r uns die einfachste Methode war, die Berechnung zu parallelisieren. Wie bei der Arbeit von Christen und Solenthaler erkl�rt, w�re Open-MPI die effizienteste Weise, um solch ein Problem zu l�sen. Da hier nur die Resultate des Algorithmus von Interesse waren und wir relativ kleine Probleme gel�st haben, hat es sich nicht gelohnt den Mehraufwand f�r Open-MPI zu betreiben. Die Rechenzeit war absolut im Rahmen.
		
		Wie schon beim Kugelsternhaufen erw�hnt, ist nun diese Open-MP Parallelisierung einfach zu realisieren, wie folgendes Beispiel zeigt. 
		
\begin{code}
// Parallelisierung einer for-Schleife
	int numthreads = 32;
	#pragma omp parallel for num_threads(numthreads)
		for (i = 0; i < dim; i++)
		{ ...
\end{code}

	\subsubsection{Dateiformate}
	
	Um mit der Datenflut umgehen zu k�nnen, musste ebenfalls eine L�sung entwickelt werden. Als Eingabedatei wurde ein Bild im FITS Format verwendet. Da es jedoch einfacher schien wurde w�hrend des Programmes mit .csv Files gearbeitet. Dabei war der technische Aufwand gering und die Resultate konnten schnell kontrolliert werden. Aus diesen .csv Dateien kann mit GNU-Plot ein Bild erstellt werden. Optional dienen diese als Vorlage f�r einen kurzen Film. 

\section{Visualisierung}
	
	\subsection{Prinzip}
			
		Ziel der Visualisierung ist, die einzelnen Schritte einer Berechnung mit der Greenschen Funktion zu zeigen. Bei jedem Schritt kommen zus�tzliche Werte zum vorherigen Schritt hinzu. Die Bilder aus \fref{fig:dreiSchritte} stellen drei solche Schritte dar, wie wir sie in unserer Arbeit definiert haben.
		
		W�rden wir jeweils die L�sungen f�r alle Werte $u_{ij}$ einzeln berechnen und sie dann addieren, m�ssten wir diese Schritte schlussendlich noch Visuell darstellen. Der Aufwand hierf�r ist aber betr�chtlich.
		Da die Darstellung im Vordergrund steht, ist das Verfahren zur Berechnung irrelevant f�r das Endergebnis. Um die Rechnung zu beschleunigen, haben wir uns f�r einen anderen Algorithmus entschieden, welcher uns erlaubt so viele Werte $u_{ij}$ wie n�tig miteinander zu l�sen.
		
		Wir werden dementsprechend immer das gleiche Problem l�sen mit jeweils zus�tzlichen Werten $u$. Jeder dieser Schritte hat ein Bild mit den vorherigen plus den neuen Werten zu folge. Zusammengestellt ergeben diese Bilder eine anschauliche Visualisierung der Greenschen Funktion.
	
	
	\subsection{Darstellungsm�glichkeiten}
	
		Welche Werte wir jeweils in einem Schritt dazu nehmen ist noch offen. Man k�nnte beispielsweise die Fl�che von oben links bis unten rechts durchgehen und immer ein oder mehrere Punkte dazu nehmen. Es ist auch vorstellbar diese Punkte zuf�llig auszuw�hlen, bis alle Punkte berechnet wurden. Der Nachteil dieser Verfahren ist, dass sie nicht viel weniger aufw�ndig sind, als das Verfahren, welches alle Punkte einzeln berechnet. Wenn beispielsweise eine Fl�che bzw. Bild mit $500 \times 500$ Werten berechnet werden soll, muss das Problem 250\,000 mal vollst�ndig gel�st werden. Es w�ren auch 250\,000 Bilder vorhanden, eine Datenflut, die unser Auge wohl kaum bew�ltigen kann. Es ist fraglich, wie sinnvoll eine solche Visualisierung ist.
		
		Um das zu verhindern, haben wir uns entschieden in der Mitte einer solchen Fl�che zu beginnen und dann Schrittweise, Ring f�r Ring, nach aussen zugehen.

	\begin{figure}
			\centering
		\begin{tabular}{ccc}
			\includegraphics[width=5.3cm]{./images/resultate/grfl/step0057.png}
			& \includegraphics[width=5.3cm]{./images/resultate/grfl/step0111.png}
			& \includegraphics[width=5.3cm]{./images/resultate/grfl/step0144.png}
		\end{tabular}		
			\caption{Die Schritte 57, 111 und 144 auf einer Fl�che, die durch $500 \times 500$ Werte dargestellt wird.}
			\label{fig:dreiSchritte}
	\end{figure}
	
		In jedem dieser drei Schritte werden nur die roten Punkte in die Berechnung ber�cksichtigt, die anderen werden gleich Null gesetzt. Jeder Schritt wir vollst�ndig gel�sst, bis der maximale Fehler eine Grenze unterschreitet.
	
	\subsection{Technische Umsetzung}
	
		Vor jeder dieser Berechnungen wird ein Vektor $f$, der unsere Fl�che repr�sentiert, so manipuliert, dass der helle �ussere Bereich gleich Null gesetzt wird. Danach wird die Berechnung, die in jedem Schritt identisch ist, durchgef�hrt.
		
		Wir speichern den berechneten Vektor $u$ als Matrix in einem .csv File mit einer Genauigkeit von vier Nachkommastellen ab, was f�r eine Visualisierung gen�gt. Anfangs benutzten wir MATLAB um aus dem .csv-File eine anst�ndige Visualisierung zu erhalten. Der Aufwand war gering und die Resultate waren schnell kontrollierbar.
		
		\lstinputlisting{./csvToMatrixAndPlot.m}
		
		Um das zu automatisieren benutzen wir sp�ter Gnuplot, welches sich mit C-Code ohne Probleme ansprechen l�sst und uns die Bilder automatisch mit dem vorgegebenen Einstellungen erstellt.
		
		Gnuplot ist ein Kommandozeilen basiertes OpenSource Tool, das aber auch mit GUI erh�ltlich ist. �ber eine Pipe ist es m�glich alle Einstellungen vorzunehmen. Dadurch kann aus einer .csv Datei ein Bild im .png Format generiert werden.
		
		\lstinputlisting{./gnuplot.c}
	
	\subsection{Generieren der Bilder}
	
		Die Bilder werden nachdem alle Einzelschritte berechnet wurden, aus den Dateien wiederum parallelisiert in einer \verb|for|-Schleife zu .png Bilddateien verarbeitet. Da der letzte Schritt das Maximum bzw. Minimum aller Schritte enth�lt, sofern wir nur Potentiale mit gleichen Vorzeichen haben, wird die Skalierung der $z$-Achse direkt aus diesen Daten berechnet. Dies garantiert und eine optimale Darstellung. 
		
		Wir haben jeweils zwei verschiedene Ansichten erstellt: Eine normal Ansicht, d.h ein 3D-Plot wie er z.B. in \fref{fig:201_1} zusehen ist und eine Ansicht von oben mit H�henlinien. Aus den Bilddateien wird am Schluss optional noch ein Video erstellt.
		
\section{Probleme}

	\subsection{Konvergenz}
	
		Wir haben anfangs den HSR-Schriftzug als FITS-File in eine Matrix 
		eingelesen und diesen als Vektor $f$ abgespeichert. Der maximale Wert des L�sungsvektors $u$ betrug anfangs nur 255, stieg nach einigen hundert Iterationen auf �ber 3000 an und die Funktion nahm die Form eines Haufens an. Wir hatten mit einem anderem Resultat gerechnet und �berpr�ften unseren Algorithmus. Wir entdeckten keinen Fehler in unserem Algorithmus und konnten die Werte mit MATLAB verifizieren. Die n�chste Frage war, ob der Gauss-Seidel Algorithmus konvergiert. Wir berechneten den Spektralradius der Matrix $A$ gem�ss Definition\;5.1 aus dem Skript HPC\cite{mueller:hpcseminar}:
	
		\begin{eqnarray}
			A = M+N\\
			\varrho(M^{-1}N)<1
		\end{eqnarray}
		
		Die Gr�sse der Matrix $A$ nimmt mit $n^4$ zu, weshalb die Berechnung der Spektralradien mit MATLAB sehr schnell rechenaufw�ndig wird. Darum haben wir nur Spektralradien mit kleinem $n$ berechnet.
		
		\begin{table}
			\begin{tabular}{cc}
				n & Specktralradius $\varrho$\\\hline
				5 & 0.7500\\
				10 & 0.92063\\
				15 & 0.96194\\
				20 & 0.97779\\
				25 & 0.98547\\
				40 & 0.99414
			\end{tabular}
			\centering
			\caption{Spaktralradien der Matrix $A$ f�r eine Matrix $f$ mit der Gr�sse $n\times n$}
		\end{table}
	
		Einerseits sind alle Spaktralradien kleiner Eins, was gut ist, andererseits sind sie sehr nahe bei Eins, was erkl�rt wieso unser Algorithmus so langsam konvergiert.
		
		Wir hatten sp�ter Erfolge, wenn wir die Anfangswerte des Vektors $f$ klein w�hlten $(f_{ij}<1)$. So waren die Fehler von Anfang an kleiner und wir bekamen brauchbare Werte.
	
	\subsection{Datenflut}
	
		Bei Platten mit feiner Diskreditierung mussten wir darauf achten, dass wir nicht zu viele Daten abspeichern. Wir mussten das Programm so umschreiben, dass wir einstellen konnten wie viele Bilder wir schlussendlich wollten. Anhand von diesem Einstellungen wurde nur die ben�tigten .csv und .png Dateien abgespeichert. Das Programm wurde dadurch automatisch auch gleich viel schneller. 
		
		\begin{shadedSmaller}
			Als Beispiel:
			
			Wir wollen eine Platte mit  $40 \times 40$ Werten berechnen. Um die Konvergenz beobachten zu k�nnen, speichern wir nach jeder Iteration die Werte in eine .csv Datei ab. Es sind etwa 3000 bis 4000 Iterationen n�tig bis sich die Werte nicht mehr ver�ndern. Das ist alles ohne Problem m�glich.
			
			Wir wollen mehr, eine Platte mit $500 \times 500$ Werten. Um hier eine Konvergenz zu beobachten, sind etwa 160\,000 bis 180\,000 Iterationen n�tig, wie wir sp�ter herausgefunden haben. Das Ganze ist nat�rlich stark vom Ausgangsbild abh�ngig.

			Eine .csv Datei ist dann etwa 1.8\,MB gross, d.h. wir wollen ca. 300\,GB abspeichern. Dies ist m�glich sofern gen�gen Platz vorhanden ist. Das eigentliche  Problem ist aber, dass das Programm nur noch mit Daten speichern besch�ftigt ist und kaum noch Zeit zum Rechnen hat.	
		\end{shadedSmaller}
	
	\subsection{geeigneter Vektor $f$ finden}\label{sec:geeignetesF}
	
	
		Ein ganz anderes Problem war, ein geeignetes Bild f�r die Platte zu finden. Es sollte die �berlagerung der Werte gut zeigen. Wir versuchten verschiedene Muster, von einem Schriftzug, einzelne Buchstaben bis zu einzelnen Punkten. Bew�hrt hat sich eine Spirale. Da wir von innen nach aussen gehen, kommen immer wieder neue Werte an verschiedenen Seiten hinzu. Da schlussendlich nur eine geringe Anzahl Punkte vorhanden sind, wirkt die Visualisierung nicht �berladen und ist �bersichtlich.
		
\section{Resultate}

	Wir haben mehrere Platten mit unterschiedlichen Potentialverteilungen und mit unterschiedlicher Anzahl Diskretisierungspunkten berechnet. Je nach Ausgangsbild, welches unsere Potentialverteilung darstellt, ist die Superposition mehr oder weniger gut erkennbar. Wie in \fref{sec:geeignetesF} erw�hnt, hat sich eine Spirale von Punkten bew�hrt, die wir hier anhand vier verschiedenen Schritten zeigen wollen. Das erste Bild ist jeweils die Fl�che f�r den jeweiligen Schritt. Alle Punkte haben das selbe Potential. Pro Bild f�hrten wir 180\,000 Iterationen durch.

	\newlength\breite
	\setlength\breite{8cm}
	\begin{figure}
			\centering
		\begin{tabular}{cc}
			\includegraphics[width=\breite]{./images/resultate/grfl/step0057.png}
			& \includegraphics[width=\breite]{./images/resultate/grfl/step0111.png}\vspace{1cm}\\
			\includegraphics[width=\breite]{./images/resultate/cp/step0057.png}
			& \includegraphics[width=\breite]{./images/resultate/cp/step0111.png}\vspace{1cm}\\
			\includegraphics[width=\breite]{./images/resultate/np/step0057.png}
			& \includegraphics[width=\breite]{./images/resultate/np/step0111.png}
		\end{tabular}		
			\caption{Visualisierung von Green's Function: Schritte 75 und 111 }
	\end{figure}
	
	\begin{figure}
			\centering
		\begin{tabular}{cc}
			\includegraphics[width=\breite]{./images/resultate/grfl/step0144.png}
			& \includegraphics[width=\breite]{./images/resultate/grfl/step0175.png}\vspace{1cm}\\
			\includegraphics[width=\breite]{./images/resultate/cp/step0144.png}
			& \includegraphics[width=\breite]{./images/resultate/cp/step0175.png}\vspace{1cm}\\
			\includegraphics[width=\breite]{./images/resultate/np/step0144.png}
			& \includegraphics[width=\breite]{./images/resultate/np/step0175.png}
		\end{tabular}		
			\caption{Visualisierung von Green's Function: Schritte 144 und 175 }
	\end{figure}
\section{Visualisierung}
	
	Wir speichern die erhaltene Matrix in einem .csv File mit einer Genauigkeit von vier Nachkommastellen ab, was absolut hinreichen f�r eine Visualisierung ist. Anfangs benutzten wir MATLAB um aus dem CSV-File eine anst�ndige Visualisierung zu erhalten. Der Aufwand war gering und die Resultate waren schnell kontrollierbar.
	
	Um das zu automatisieren benutzen wir sp�ter Gnuplot, welches sich mit C-Code ohne Probleme ansprechen l�sst und uns die Bilder automatisch mit dem vorgegebenen Einstellungen erstellt.
	
	Gnuplot ist ein Kommandozeilen basiertes OpenSource Tool, das aber auch mit GUI erh�ltlich ist. �ber eine Pipe lassen sich lassen sich alle Einstellungen vornehmen und aus dem .CSV-File ein PNG generieren.
	
	Mit der Green-Funtion k�nnen bei einer partiellen DGL 2. Ordnung einzelne Punkte berechnet werden. Um das zu visualisieren, berechnen wir einzelne Ringe bzw. Quadrate vom Zentrum ausgehen nach aussen. Man k�nnte nat�rlich auch jeder Punkt einzeln dazu nehmen, was mehr Berechnungen erfordert.
	

	
	Diese einzelnen Schritte werden vollst�ndig berechnet und dann als .csv abgespeichert. Die Bilder werden nachdem alle Einzelschritte berechnet wurden aus den Dateien wiederum parallelisiert in einer \verb|for|-Schleife zu .png Bilddateien verarbeitet. Die Skalierung der z-Achse wird direkt aus den Daten des letzten Schrittes berechnet, was uns eine optimale Darstellung garantiert. Aus den Bilddateien wird am Schluss optional noch ein Video erstellt.
		
\section{Algorithmus}

	Da wir aus einer Matrix von Werten eine andere, gleich grosse  Matrix mit weiteren Werten berechnen, m�ssen wir zuerst diese Matrix in einen Vektor zerlegen.
	
	\begin{equation}
		\dunderline{A}\cdot \dunderline{x} \ne \dunderline{f} \qquad\Rightarrow\qquad \dunderline{A}\cdot \underline{x} = \underline{f}
		\label{eq:gleichung}
	\end{equation}
	
	Dabei haben wir bei beiden Matrizen die Spalten untereinander gereiht und so je einen Vektor mit der L�nge $n^2$ erhalten. Mathematisch macht es keinen Unterschied, solange wir am Schluss den L�sungsvektor wieder auf die selbe Art und Weise zu einer Matrix zusammenf�gen.
	
\[
	A=\left(
	\begin{array}{ccccc|ccccc|c|ccccc}
	    -4&     1&     0&\cdots&     0 &     1&     0&     0&\cdots&     0 &\cdots &      &      &      &      &      \\
	     1&    -4&     1&\cdots&     0 &     0&     1&     0&\cdots&     0 &\cdots &      &      &      &      &      \\
	     0&     1&    -4&\cdots&     0 &     0&     0&     1&\cdots&     0 &\cdots &      &      &     0&      &      \\
	\vdots&\vdots&\vdots&\ddots&\vdots &\vdots&\vdots&\vdots&\ddots&\vdots &       &      &      &      &      &      \\
	     0&     0&     0&\cdots&    -4 &     0&     0&     0&\dots &     1 &\cdots &      &      &      &      &      \\
	\hline
	     1&     0&     0&\cdots&     0 &    -4&     1&     0&\dots &     0 &\cdots &      &      &      &      &      \\
	     0&     1&     0&\cdots&     0 &     1&    -4&     1&\dots &     0 &\cdots &      &      &      &      &      \\
	     0&     0&     1&\cdots&     0 &     0&     1&    -4&\dots &     0 &\cdots &      &      &     0&      &      \\
	\vdots&\vdots&\vdots&\ddots&\vdots &\vdots&\vdots&\vdots&\ddots&\vdots &       &      &      &      &      &      \\
	     0&     0&     0&\cdots&     1 &     0&     0&     0&\cdots&    -4 &\cdots &      &      &      &      &      \\
	\hline
	\vdots&\vdots&\vdots&      &\vdots &\vdots&\vdots&\vdots&      &\vdots &\ddots &\vdots&\vdots&\vdots&      &\vdots\\
	\hline
	      &      &      &      &       &      &      &      &      &       &\cdots &    -4&     1&     0&\cdots&     0\\
	      &      &      &      &       &      &      &      &      &       &\cdots &     1&    -4&     1&\cdots&     0\\
	      &      &     0&      &       &      &      &     0&      &       &\cdots &     0&     1&    -4&\cdots&     0\\
	      &      &      &      &       &      &      &      &      &       &       &\vdots&\vdots&\vdots&\ddots&\vdots\\
	      &      &      &      &       &      &      &      &      &       &\cdots &     0&     0&     0&\cdots&    -4\\
	\end{array}
	\right) 
	\]
	
	Die Matrix $A$ enth�lt die Koeffizienten f�r die partielle Differentialgleichung der zweiten Ordnung. Sie hat die Gr�sse $n^2 \times n^2$. Diese Matrix auf dem Heap zu allozieren w�re f�r mittelgrosse Matrizen schon nicht mehr m�glich.\cite{mueller:hpcseminar}
	
	
	
	Der ben�tigte Speicher f�r eine Matrix $f$ mit der Gr�sse $500 \times 500$ und \verb|float| Werten (4 Byte) w�re
	
	\begin{equation}
		500^2 \times 500^2 \cdot 4\;\mathrm{Byte} = 232.8\;\mathrm{GByte}
	\end{equation}
	
	Mit $n = 1000$ sogar 3.64\;TByte, also definitiv zu viel. Dies ist aber auch gar nicht n�tig. Es m�ssen jeweils nur die Nachbarelemente von $f_k$ addiert und durch einen konstanten Faktor geteilt werden, die Randelemente sind dabei Null zu setzen.

	\begin{eqnarray}
		f_k = U_{k-1}+U_{k+1}+U_{k-(n-1)}+U_{k+(n-1)}-4U_k
	\end{eqnarray}
	
	Da die diskrete partielle Ableitung jeweils nur die Elemente ober/unterhalb und links/recht des aktuellen Elementes in die Rechnung mit einbezieht, m�ssen mindestens $n$ Iterationen durchgef�hrt werden, damit sich das Potential �ber die ganze Ebene verteilt.
	
	\subsection{Parallelisierung}
		
		Um das Gleichungssystem (\ref{eq:gleichung}) zu l�sen, haben wir den Gauss-Seidel Algorithmus benutzt. Bei diesem Algorithmus ist, wie wir wissen, die aktuelle Zeile von der vorherigen abh�ngig. Bei der Parallelisierung wird das Gleichungssystem an verschiedenen Stellen zu l�sen begonnen. Das ist zwar nicht so effizient, wie ein "'normaler"' Gauss-Seidel, wird aber durch die Parallelisierung schneller.
		
		Die Abweichung ist bei den ersten Schritten am gr�ssten, und wird bei jeder Iteration kleiner. Als wie die ersten Berechnungen durch gef�hrt hatten, fiel uns auf, dass die einzelnen Threads am Anfang gut als Linien von Spitzen sichtbar sind (\fref{fig:201_1}).
		
		\begin{figure}[h]
			\centering
			\includegraphics[width = 15cm]{./images/step001}
			\caption{Berechnung von einer $201 \times 201$ Fl�che mit 32 Threads nach dem ersten Iterationsschritt}
			\label{fig:201_1}
		\end{figure}
		
		
		Wir haben uns f�r OpenMP entschieden, da wir ein Grosses Problem haben, welches immer auf den selben Speicher zugreift. Wie schon beim Kugelsternhaufen erw�hnt, ist die Parallelisierung einfach zu realisieren:
		
\begin{code}
	int numthreads = 32;
	#pragma omp parallel for num_threads(numthreads)
		for (i = 0; i < dim; i++)
		{ ...
\end{code}
		
		
		
		


%\section{FITS Dateiformat}\label{sec:FITS}

Flexible Image Transport System ist die ausgeschriebene Bezeichnung f�r das Wort FITS. Es ist ein quelloffenes Dateiformat, womit Daten verlustlos abgespeichert werden k�nnen. Die Nasa entwickelte es und in der Astronomie hat es sich als Standardformat etabliert.  Informationen k�nnen in verschiedenen Layern abgelegt werden. Das bedeutet, ein Gitterpunkt in einem Datenfeld kann unterschiedliche Informationen enthalten. Angenehm ist ausserdem, dass die Werte direkt in Flisskommadarstellung abgespeichert werden k�nnen.\cite{nasa:fits}

Wir haben es in diesem Projekt als Eingabebild verwendet, weil wir mit einem einfachen Bildbearbeitungsprogramm Testbilder herstellen konnten.  
  %schon im algorithmusteil behandelt
\section{Probleme}
	
	Die Fl�che $f$ kann alle m�glichen Werte als Anfangsbedingung haben. Wir haben anfangs den  HSR-Schriftzug als FITS-File in eine Matrix 
	eingelesen. Der maximale Wert betrug anfangs nur 255,
	stieg nach einigen hundert Iterationen auf �ber 3000 an und die Funktion nahm die Form eines Haufen an. Wir hatten mit einem anderem Resultat gerechnet und �berpr�ften unseren Algorithmus. Wir konnten keinen Fehler entdecken und konnten die Werte mit MATLAB verifizieren. Die n�chste Frage war, ob der Gauss-Seidel konvergiert. Wir berechneten den Spektralradius der Matrix $A$ gem�ss Definition\;5.1 aus dem Skript HPC\cite{mueller:hpcseminar}:
	
	\begin{eqnarray}
		A = M+N\\
		\varrho(M^{-1}N)<1
	\end{eqnarray}
	
	Da die Berechnung schnell sehr rechenaufwendig wird, man beachte, dass die Gr�sse der Matrix $A$ mit $n^4$ zunimmt, haben wir Matrizen mit kleinem $n$ berechnet. 	
	
	\begin{table}[h]
		\begin{tabular}{cc}
			n & Specktralradius $\varrho$\\\midrule
			5 & 0.7500\\
			10 & 0.92063\\
			15 & 0.96194\\
			20 & 0.97779\\
			25 & 0.98547\\
			40 & 0.99414
		\end{tabular}
		\centering
		\caption{Spaktralradien der Matrix $A$ f�r eine Matrix $f$ mit der Gr�sse $n\times n$}
	\end{table}

	Einerseits sind alle Werte kleiner Eins, was gut ist, andererseits sind die Werte sehr nahe bei Eins, was erkl�rt wieso unser Algorithmus so langsam konvergiert.
	
	Wir hatten sp�ter Erfolge, wenn wir die Anfangswert der Matrix $f$ klein w�hlten $(f_{ij}<1)$. So waren die Fehler von Anfang an kleiner und wir bekamen brauchbare Werte.
	
	Bei gr�sseren Matrizen mussten wir darauf achten, dass wir nicht zu viele Daten abspeichern. Wir mussten das Programm so umschreiben, dass wir einstellen konnten wie viele Bilder wir schlussendlich wollten. Anhand von diesem Einstellungen wurde nur die ben�tigten .csv und .png Dateien abgespeichert. Das Programm wurde dadurch automatisch auch gleich viel schneller. 
\section{Resultate}

In unserem Schlussresultat ist die Superposition gut erkennbar. Ausgangsbild war eine Spirale auf der sich einige Punkte befinden mit der gleichen Stromst�rke. Pro Bild f�hrten wir 180\,000 Iterationen durch.

\newlength\breite
\setlength\breite{8cm}

\begin{figure}
		\centering
	\begin{tabular}{cc}
		\includegraphics[width=\breite]{./images/resultate/step0111.png}
		& \includegraphics[width=\breite]{./images/resultate/step0112.png}\\
		\includegraphics[width=\breite]{./images/resultate/step0113.png}
		& \includegraphics[width=\breite]{./images/resultate/step0138.png}
	\end{tabular}		
		\caption{Visualisierung von Green's Function: Schritte 111, 112, 113, 138}
\end{figure}

\begin{figure}
		\centering
	\begin{tabular}{cc}
		\includegraphics[width=\breite]{./images/resultate/step0144.png}
		& \includegraphics[width=\breite]{./images/resultate/step0149.png}\\
		\includegraphics[width=\breite]{./images/resultate/step0162.png}
		& \includegraphics[width=\breite]{./images/resultate/step0175.png}
	\end{tabular}		
		\caption{Visualisierung von Green's Function: Schritte 144, 149, 162, 175 }
\end{figure}


\printbibliography
\listoffigures

\end{document}