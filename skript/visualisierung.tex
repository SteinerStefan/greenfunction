\section{Visualisierung}
	
	Wir speichern die erhaltene Matrix in einem .csv File mit einer Genauigkeit von vier Nachkommastellen ab, was f�r eine Visualisierung absolut hinreichen ist. Anfangs benutzten wir MATLAB um aus dem CSV-File eine anst�ndige Visualisierung zu erhalten. Der Aufwand war gering und die Resultate waren schnell kontrollierbar.
	
	\lstinputlisting{./csvToMatrixAndPlot.m}
	
	Um das zu automatisieren benutzen wir sp�ter Gnuplot, welches sich mit C-Code ohne Probleme ansprechen l�sst und uns die Bilder automatisch mit dem vorgegebenen Einstellungen erstellt.
	
	Gnuplot ist ein Kommandozeilen basiertes OpenSource Tool, das aber auch mit GUI erh�ltlich ist. �ber eine Pipe lassen sich lassen sich alle Einstellungen vornehmen und aus dem .CSV-File ein PNG generieren.
	
	\lstinputlisting{./gnuplot.c}
	
	Mit der Green-Funtion k�nnen bei einer partiellen DGL 2. Ordnung einzelne Punkte berechnet werden. Um das zu visualisieren, berechnen wir einzelne Ringe bzw. Quadrate vom Zentrum ausgehen nach aussen. Man k�nnte nat�rlich auch jeder Punkt einzeln dazu nehmen, was mehr Berechnungen erfordert.
	
		\[
			f=\left(
			\begin{array}{ccccc}
			    f_{11}& f_{12}& f_{13}& f_{14}& f_{15}\\
				f_{21}& f_{22}& f_{23}& f_{24}& f_{25}\\
				f_{31}& f_{32}& \red{f_{33}}& f_{34}& f_{35}\\
				f_{41}& f_{42}& f_{43}& f_{44}& f_{45}\\
				f_{51}& f_{52}& f_{53}& f_{54}& f_{55}\\
			\end{array}
			\right) \Rightarrow
			\left(
			\begin{array}{ccccc}
			    f_{11}& f_{12}& f_{13}& f_{14}& f_{15}\\
				f_{21}& \red{f_{22}}& \red{f_{23}}& \red{f_{24}}& f_{25}\\
				f_{31}& \red{f_{32}}& \red{f_{33}}& \red{f_{34}}& f_{35}\\
				f_{41}& \red{f_{42}}& \red{f_{43}}& \red{f_{44}}& f_{45}\\
				f_{51}& f_{52}& f_{53}& f_{54}& f_{55}\\
			\end{array}\right)
			\Rightarrow
			\left(\red{
			\begin{array}{ccccc}
			    f_{11}& f_{12}& f_{13}& f_{14}& f_{15}\\
				f_{21}& f_{22}& f_{23}& f_{24}& f_{25}\\
				f_{31}& f_{32}& f_{33}& f_{34}& f_{35}\\
				f_{41}& f_{42}& f_{43}& f_{44}& f_{45}\\
				f_{51}& f_{52}& f_{53}& f_{54}& f_{55}\\
			\end{array}}
			\right)
			\]

	
	Diese einzelnen Schritte werden vollst�ndig berechnet und dann als .csv abgespeichert. Die Bilder werden nachdem alle Einzelschritte berechnet wurden aus den Dateien wiederum parallelisiert in einer \verb|for|-Schleife zu .png Bilddateien verarbeitet. Die Skalierung der z-Achse wird direkt aus den Daten des letzten Schrittes berechnet, was uns eine optimale Darstellung garantiert. Aus den Bilddateien wird am Schluss optional noch ein Video erstellt.
		