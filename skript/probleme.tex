\section{Probleme}
	
	Die Fl�che $f$ kann alle m�glichen Werte als Anfangsbedingung haben. Wir haben anfangs den  HSR-Schriftzug als FITS-File in eine Matrix 
	eingelesen. Der maximale Wert betrug anfangs nur 255,
	stieg nach einigen hundert Iterationen auf �ber 3000 an und die Funktion nahm die Form eines Haufen an. Wir hatten mit einem anderem Resultat gerechnet und �berpr�ften unseren Algorithmus. Wir konnten keinen Fehler entdecken und konnten die Werte mit MATLAB verifizieren. Die n�chste Frage war, ob der Gauss-Seidel konvergiert. Wir berechneten den Spektralradius der Matrix $A$ gem�ss Definition\;5.1 aus dem Skript HPC\cite{mueller:hpcseminar}:
	
	\begin{eqnarray}
		A = M+N\\
		\varrho(M^{-1}N)<1
	\end{eqnarray}
	
	Da die Berechnung schnell sehr rechenaufwendig wird, man beachte, dass die Gr�sse der Matrix $A$ mit $n^4$ zunimmt, haben wir Matrizen mit kleinem $n$ berechnet. 	
	
	\begin{table}[h]
		\begin{tabular}{cc}
			n & Specktralradius $\varrho$\\\midrule
			5 & 0.7500\\
			10 & 0.92063\\
			15 & 0.96194\\
			20 & 0.97779\\
			25 & 0.98547\\
			40 & 0.99414
		\end{tabular}
		\centering
		\caption{Spaktralradien der Matrix $A$ f�r eine Matrix $f$ mit der Gr�sse $n\times n$}
	\end{table}

	Einerseits sind alle Werte kleiner Eins, was gut ist, andererseits sind die Werte sehr nahe bei Eins, was erkl�rt wieso unser Algorithmus so langsam konvergiert.
	
	Wir hatten sp�ter Erfolge, wenn wir die Anfangswert der Matrix $f$ klein w�hlten $(f_{ij}<1)$. So waren die Fehler von Anfang an kleiner und wir bekamen brauchbare Werte.
	
	Bei gr�sseren Matrizen mussten wir darauf achten, dass wir nicht zu viele Daten abspeichern. Wir mussten das Programm so umschreiben, dass wir einstellen konnten wie viele Bilder wir schlussendlich wollten. Anhand von diesem Einstellungen wurde nur die ben�tigten .csv und .png Dateien abgespeichert. Das Programm wurde dadurch automatisch auch gleich viel schneller. 