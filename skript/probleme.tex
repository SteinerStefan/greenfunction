\section{Probleme}
	\subsection{Konvergenz}
	
		Die Fl�che $f$ kann alle m�glichen Werte als Anfangsbedingung haben. Wir haben anfangs den  HSR-Schriftzug als FITS-File in eine Matrix 
		eingelesen. Der maximale Wert des $x$ Vektors betrug anfangs nur 255,
		stieg nach einigen hundert Iterationen auf �ber 3000 an und die Funktion nahm die Form eines Haufens an. Wir hatten mit einem anderem Resultat gerechnet und �berpr�ften unseren Algorithmus. Wir entdeckten keinen Fehler in unserem Algorithmus und konnten die Werte mit MATLAB verifizieren. Die n�chste Frage war, ob der Gauss-Seidel konvergiert. Wir berechneten den Spektralradius der Matrix $A$ gem�ss Definition\;5.1 aus dem Skript HPC\cite{mueller:hpcseminar}:
	
		\begin{eqnarray}
			A = M+N\\
			\varrho(M^{-1}N)<1
		\end{eqnarray}
		
		Da die Berechnung schnell sehr rechenaufwendig wird, man beachte, dass die Gr�sse der Matrix $A$ mit $n^4$ zunimmt, haben wir Matrizen mit kleinem $n$ berechnet. 	
		
		\begin{table}[h]
			\begin{tabular}{cc}
				n & Specktralradius $\varrho$\\\midrule
				5 & 0.7500\\
				10 & 0.92063\\
				15 & 0.96194\\
				20 & 0.97779\\
				25 & 0.98547\\
				40 & 0.99414
			\end{tabular}
			\centering
			\caption{Spaktralradien der Matrix $A$ f�r eine Matrix $f$ mit der Gr�sse $n\times n$}
		\end{table}
	
		Einerseits sind alle Werte kleiner Eins, was gut ist, andererseits sind die Werte sehr nahe bei Eins, was erkl�rt wieso unser Algorithmus so langsam konvergiert.
		
		Wir hatten sp�ter Erfolge, wenn wir die Anfangswert der Matrix $f$ klein w�hlten $(f_{ij}<1)$. So waren die Fehler von Anfang an kleiner und wir bekamen brauchbare Werte.
	
	\subsection{Datenflut}
	
		Bei gr�sseren Matrizen mussten wir darauf achten, dass wir nicht zu viele Daten abspeichern. Wir mussten das Programm so umschreiben, dass wir einstellen konnten wie viele Bilder wir schlussendlich wollten. Anhand von diesem Einstellungen wurde nur die ben�tigten .csv und .png Dateien abgespeichert. Das Programm wurde dadurch automatisch auch gleich viel schneller. 
		
		\begin{shadedSmaller}
			Als Beispiel:
			
			Wir wollen eine $40 \times 40$ Fl�che berechnen. Um die Konvergenz beobachten zu k�nnen, speichern wir nach jeder Iteration die Werte in eine .csv Datei ab. Es sind etwa 3000 bis 4000 Iterationen n�tig bis sich die Werte nicht mehr ver�ndern. Alles ohne Problem m�glich.
			
			Wir wollen mehr, eine Fl�che mit $500 \times 500$ Punkten. Um hier eine Konvergenz zu beobachten sind, wie wir sp�ter herausgefunden haben, etwa 160\,000 bis 180\,000 Iterationen n�tig. Das Ganze ist nat�rlich stark vom Ausgangsbild abh�ngig.

			Eine .csv Datei ist dann etwa 1.8\,MB gross, d.h. wir wollen ca. 300\,GB abspeichern. Dies ist m�glich sofern gen�gen Platz vorhanden ist. Das eigentliche  Problem ist aber, dass das Programm nur noch mit Daten speichern besch�ftigt ist und kaum noch Zeit zum Rechnen hat.	
		\end{shadedSmaller}

	
	\subsection{geeignete Fl�che $f$ finden}\label{sec:geeignetesF}
	
	
		Ein ganz anderes Problem war, ein geeignetes Bild f�r die Fl�che $f$ zu finden. Es sollte die �berlagerung der Werte gut zeigen. Wir versuchten verschiedene Muster, von einem Schriftzug, einzelne Buchstaben bis zu einzelnen Punkten. Bew�hrt hat sich eine Spirale. Da wir von innen nach aussen gehen, kommen immer wieder neue Punkte auf verschiedenen Seiten hinzu. Da schlussendlich nur eine geringe Anzahl Punkte vorhanden sind, wirkt die Visualisierung nicht �berladen und ist �bersichtlich.