\section{Algorithmus}

	Da wir aus einer Matrix von Werten eine andere, gleich grosse  Matrix mit weiteren Werten berechnen, m�ssen wir zuerst diese Matrix in einen Vektor zerlegen.
	
	\begin{equation}
		\dunderline{A}\cdot \dunderline{X} \ne \dunderline{V} \qquad\Rightarrow\qquad \dunderline{A}\cdot \underline{x} = \underline{v}
		\label{eq:gleichung}
	\end{equation}
	
	Dabei haben wir bei beiden Matrizen die Spalten untereinander gereiht und so je einen Vektor mit der L�nge $n^2$ erhalten. Mathematisch macht das keinen Unterschied, solange wir am Schluss den L�sungsvektor wieder auf die selbe Art und Weise zu einer Matrix zusammenf�gen. Wir haben nun ein ganz normales lineares Gleichungssystem.
	
	Die Matrix $A$ enth�lt die Koeffizienten f�r die partielle Differentialgleichung 2. Ordnung, wie wir sie in der Vorlesung hergeleitet haben. Sie hat die Gr�sse $n^2 \times n^2$. Diese Matrix auf dem heap zu allozieren w�re f�r mittelgrosse Matrizen $V$ schon nicht mehr m�glich.
	
	Der ben�tigte Speicher f�r eine Matrix $V$ mit der Gr�sse $500 \times 500$ und float Werten (4 Byte) w�re
	
	\begin{equation}
		500^2 \times 500^2 \cdot 4\;\mathrm{Byte} = 232.8\;\mathrm{GByte}
	\end{equation}
	
	Mit $n = 1000$ sogar 3.64\;TByte, also definitiv zuviel. Um dieses Problem zu umgehen, errechnen wir die Koeffizienten f�r jeden Schritt einzeln, was mit kleinem Zeitaufwand m�glich ist da die Matrix $A$ d�nn besetzt ist.
	
	Da die diskrete partielle Ableitung jeweils nur die Elemente ober/unterhalb und links/recht des aktuellen Elementes in die Rechnung mit einbezieht, m�ssen mindestens $n$ Iterationen durchgef�hrt werden, damit sich das \emph{Potential} �ber die ganze Ebene verteilt.
	
	\subsection{Parallelisierung}
		
		Um das Gleichungssystem (\ref{eq:gleichung}) zu l�sen, haben wir den Gauss-Seidel Algorithmus benutzt. Bei diesem Algorithmus ist, wie wir wissen, die aktuelle Zeile von der vorherigen abh�ngig. Bei der Parallelisierung wird das Gleichungssystem an verschiedenen Stellen zu l�sen begonnen. Das ist zwar nicht so effizient, wie ein "'normaler"' Gauss-Seidel, wird aber durch die Parallelisierung schneller.
		
		Wir haben uns f�r OpenMP entschieden, da ...
		
	